\documentclass[handout]{beamer} 
\title{ITCS 532:\\ 
4. Undecidable Problems}
\date{}
\author{Rob Egrot}

\usepackage{amsmath, bbold, bussproofs,graphicx}
\usepackage{mathrsfs}
\usepackage{amsthm}
\usepackage{amssymb}
\usepackage[all]{xy}
\usepackage{multirow}
\usepackage{tikz-cd}


\newtheorem{proposition}[theorem]{Proposition}
\newcommand{\bN}{\mathbb{N}}
\newcommand{\bZ}{\mathbb{Z}}
\newcommand{\bQ}{\mathbb{Q}}
\newcommand{\bR}{\mathbb{R}}
\newcommand{\bP}{\mathbb{P}}
\newcommand{\tvs}{\textvisiblespace}
\newcommand{\ra}{\rightarrow}
\newcommand{\la}{\leftarrow}
\newcommand{\co}{\mathbf{code}}

\addtobeamertemplate{navigation symbols}{}{%
    \usebeamerfont{footline}%
    \usebeamercolor[fg]{footline}%
    \hspace{1em}%
    \insertframenumber/\inserttotalframenumber
}
\setbeamertemplate{theorems}[numbered]
\begin{document}

\begin{frame}
\titlepage
\end{frame}

\begin{frame}
\frametitle{Undecidable Problems}
\begin{itemize}
\item In the previous class we saw that there are undecidable problems.
\vspace{0.1cm}
\item This was based on a cardinality argument.
\vspace{0.1cm}
\item The set of formal languages for a finite alphabet is uncountable.
\vspace{0.1cm}
\item But the set of Turing machines for this language is countable.
\vspace{0.1cm}
\item Since every recursively enumerable language requires a TM to semidecide it, and no TM semidecides more than one language, there must be languages that are not r.e.
\vspace{0.1cm}
\item So there are undecidable problems.
\vspace{0.1cm}
\item But can we find an actual undecidable problem?
\end{itemize}
\end{frame}

\begin{frame}
\frametitle{The Halting Problem}
\begin{itemize}
\item Given a Turing machine $T$ and an input $I$, either $T(I)$ halts or it runs forever. 
\vspace{0.2cm}
\item One must happen, and both cannot be true at the same time. 
\vspace{0.2cm}
\item So the question of whether a Turing machine $T$ halts on input $I$ is a \emph{decision problem} whose instances are pairs $(T,I)$. 
\vspace{0.2cm}
\item We know that we can encode pairs $(T,I)$ of Turing machines and inputs over the alphabet $\{0,1\}$. 
\vspace{0.2cm}
\item So, the question is, can we create a Turing machine that decides this problem? 
\vspace{0.2cm}
\item Is there a TM that takes as input $\co(T,I)$ and accepts if $T(I)$ halts, and rejects if it does not (or if its input is not the code for a TM and an input)? 
\end{itemize}
\end{frame}

\begin{frame}
\frametitle{If the Halting Problem were decidable...}
\begin{itemize}
\item If a Turing machine $H$ exists that can decide the halting problem then, for one thing, it would imply that r.e. languages are recursive. 
\vspace{1cm}
\item Why? 
\end{itemize}
\end{frame}

\begin{frame}
\frametitle{Why is the Halting Problem undecidable}
\begin{itemize}
\item How can we prove the Halting Problem is undecidable?
\vspace{0.2cm}
\item We will look for a contradiction.
\vspace{0.2cm}
\item So suppose $H$ exists. I.e., suppose for any input $J$ we have $H(J)$ accepts if $J=\co(T,I)$ for some Turing machine $T$ and input $I$ such that $T(I)$ halts, and rejects otherwise. 
\vspace{0.2cm}
\item What are the consequences of this?  
\end{itemize}
\end{frame}



\begin{frame}
\frametitle{Self Reference}
\begin{itemize}
\item There are many logical paradoxes arising from `self reference'.  
\vspace{0.2cm}
\item E.g. `the set of all sets that don't contain themselves'.
\vspace{0.2cm}
\item Can we use this here?
\vspace{0.2cm}
\item Consider the machine $H'$ such that $H'(\co(T))$ accepts iff $T(\co(T))$ halts, and rejects otherwise.
\vspace{0.2cm}
\item Note that if $H$ exists, then $H'$ also exists, as we can run $H(\co(T,\co(T)))$.
\vspace{0.2cm}
\item What happens if we run $H'$ on $\co(H')$?
\vspace{0.2cm}
\item $H'$ always halts, so $H'(\co(H'))$ accepts. No problem here. 
\end{itemize}
\end{frame}

\begin{frame}
\frametitle{A Variation}
\begin{itemize}
\item Define a new machine $H''$. 
\vspace{0.2cm}
\item This machine is based on $H'$ but it halts if $T(\co(T))$ \emph{doesn't} halt, and it loops forever if $T(\co(T))$ halts. 
\vspace{0.2cm}
\item This modification is easy to make; it's like the one used to prove that recursive languages are recursively enumerable. 
\[H''(\co(T))=\begin{cases}Halt \text{ if } T(\co(T)) \text{ doesn't halt}\\ Loop\text{ }forever \text{ if } T(\co(T)) \text{ halts}\end{cases}\] 
In addition, $H''(I)$ halts if $I$ is not the coded form of a TM.
\end{itemize}
\end{frame}

\begin{frame}
\frametitle{Self Reference Again}
\begin{itemize}
\item What happens when we run $H''$ on its own code. 
\item I.e. what is $H''(\co(H''))$? 
\item According to the definition, $H''(\co(H''))$ halts if and only if $H''(\co(H''))$ doesn't halt. 
\[H''(\co(H''))=\begin{cases}Halt \text{ if } H''(\co(H'')) \text{ doesn't halt}\\ Loop\text{ }forever \text{ if } H''(\co(H'')) \text{ halts}\end{cases}\] 
\item This is a clear contradiction, so we must conclude that $H''$ cannot exist. But $H''$ is a simple modification of $H$, so $H$ cannot exist either. This gives us:

\begin{theorem}
The halting problem is undecidable.
\end{theorem} 
\end{itemize}
\end{frame}

\begin{frame}
\frametitle{The Halting Problem \emph{is} Semidecidable}
\begin{itemize}
\item The Halting Problem is semidecidable.
\vspace{0.4cm}
\item Why?
\vspace{0.4cm}
\item As a consequence, the set of recursive languages is strictly contained in the set of $r.e.$ languages.
\vspace{0.4cm}
\item Because the formal language containing strings $\co(T,I)$ (in some alphabet) such that $T(I)$ halts is r.e. but not recursive.
\end{itemize}
\end{frame}

\begin{frame}
\frametitle{Languages That Are Not R.E.}
\begin{itemize}
\item The Halting problem is semidecidable but not decidable, so its associated language is r.e. but not recursive.
\vspace{0.4cm}
\item We know from the countable vs uncountable argument that there are languages that are not r.e.
\vspace{0.4cm}
\item Can we find an example?
\vspace{0.4cm}
\item It turns out the answer is yes.
\vspace{0.4cm}
\item First some facts about recursive and r.e. languages.
\end{itemize}
\end{frame}

\begin{frame}
\frametitle{}
\begin{theorem}\label{T:rec}
If $L$ is a recursive formal language then the complement language $\bar{L}$ is also recursive.
\end{theorem}
\begin{proof}
\begin{itemize}
\item If $L$ is recursive then there's a Turing machine $T$ that accepts if $I\in L$ and rejects if $I\notin L$. 
\item To get a machine that decides $\bar{L}$ we just need to swap the `accept' and `reject' states of $T$.
\end{itemize}
\end{proof}
\end{frame}

\begin{frame}
\frametitle{}
\begin{theorem}\label{T:comps}
Let $L$ be a formal language over a finite alphabet. If $L$ is r.e. and the complement language $\bar{L}$ is also r.e. then $L$ is recursive.
\end{theorem}
\begin{proof}
\begin{itemize}
\item  If there are machines $T_1$ and $T_2$ that semidecide $L$ and $\bar{L}$ respectively then we can construct a machine $T$ to decide $L$ as follows. 
\item Given input $I$ we simulate $T_1(I)$ and $T_2(I)$. 
\item Either $I\in L$ or $I\in \bar{L}$ so one of $T_1(I)$ and $T_2(I)$ will halt. 
\item If $T_1(I)$ halts $T$ accepts, if $T_2(I)$ halts then $T$ rejects. 
\end{itemize} 
\end{proof}
\end{frame}

\begin{frame}
\frametitle{}
\begin{theorem}
Let $L_1$ and $L_2$ be formal languages over a finite alphabet.
\begin{enumerate}
\item If $L_1$ and $L_2$ are recursive then $L_1\cup L_2$ and $L_1\cap L_2$ are recursive.
\item If $L_1$ and $L_2$ are r.e. then $L_1\cup L_2$ and $L_1\cap L_2$ are r.e.
\end{enumerate}
\end{theorem}
\begin{proof}
\begin{itemize}
\item Let $T_1$ and $T_2$ be machines that decide $L_1$ and $L_2$ respectively.
\item Simulating $T_1$ and $T_2$ on any input $I$. 
\item If either $T_1(I)$ or $T_2(I)$ accepts then $I\in L_1\cup L_2$. On the other hand, if both reject then $I\notin L_1\cup L_2$. 
\item If both accept then $I\in L_1\cap L_2$, but if one rejects then $I\notin L_1\cap L_2$.
\item Similarly if $T_1$ and $T_2$ are machines that semidecide $L_1$ and $L_2$ then $I\in L_1\cup L_2$ if and only if either $T_1(I)$ or $T_2(I)$ halts, and $I\in L_1\cap L_2$ if and only if $T_1(I)$ and $T_2(I)$ both halt. 
\end{itemize}
\end{proof}
\end{frame}

\begin{frame}
\frametitle{Three languages}
\begin{definition}[$SA$ - `self accepting']
$SA$ is the language over $\{0,1\}$ that contains the codes of all Turing machines $T$ such that $T(\co(T))$ halts. 
\end{definition}
\vspace{0.4cm}
\begin{definition}[$NSA$ - `not self accepting']
$NSA$ is the language over $\{0,1\}$ that contains the codes of all Turing machines $T$ such that $T(\co(T))$ does not halt.
\end{definition}
\vspace{0.4cm}
\begin{definition}[$NSA'$]
$NSA'$ is $NSA$ together with all the strings that are not codes for Turing machines. So $NSA'=\overline{SA}$.
\end{definition}
\end{frame}

\begin{frame}
\frametitle{The Halting Problem and $SA$}
\begin{itemize}
\item $SA$ contains all the yes instances of the modified Halting Problem from earlier.
\item We showed there is no TM that decides this problem.
\item So $SA$ is not recursive (but is r.e. as we can design a machine to semidecide it).
\item So...
\end{itemize}

\begin{theorem}
$NSA'$ is not r.e.
\end{theorem}
\begin{proof}
Since $SA$ is r.e., if $NSA'$ is also r.e. then $SA$ is recursive by theorem \ref{T:comps} (as $NSA'=\overline{SA}$). But $SA$ is not recursive, so $NSA'$ cannot be r.e.
\end{proof}
\end{frame}

\begin{frame}
\frametitle{Application to $NSA$}
\begin{corollary}
$NSA$ is not r.e.
\end{corollary}
\begin{proof}
\begin{itemize}
\item By the definition of coding there is an algorithm that says whether a string $x$ is or is not the coded form of a Turing machine. 
\item So if $NSA$ was r.e. then we could combine this with the algorithm semideciding $NSA$:
\begin{enumerate}
\item Check if input string $x$ is the code of a Turing machine. If no then halt, if yes go to step 2.
\item Run algorithm semideciding $NSA$. If this halts then halt, otherwise just keep going. 
\end{enumerate} 
\item This algorithm semidecides $NSA'$, which is a contradiction because we know $NSA'$ is not r.e.
\end{itemize}
\end{proof}
\end{frame}

\begin{frame}
\frametitle{A Direct Argument for $NSA$}
We could also prove that $NSA$ is not r.e. directly, using similar logic to what we used to show the halting problem is not decidable.
\vspace{0.4cm}
\begin{theorem}\label{T:NSAalt}
$NSA$ is not r.e. (without using the fact that the modified halting problem is undecidable).
\end{theorem}
\begin{proof}
\begin{itemize}
\item Suppose there is a machine $M$ that semidecides $NSA$. 
\item So $M(\co(T))$ halts if $T(\co(T))$ does not halt, and runs forever if $T(\co(T))$ halts. 
\item So by definition $M(\co(M))$ halts if and only if $M(\co(M))$ does not halt, which is a contradiction.
\end{itemize}
\end{proof}
\end{frame}

\begin{frame}
\frametitle{Another Argument for $NSA'$}
\begin{corollary}\label{C:NSA'}
$NSA'$ is not r.e.
\end{corollary}
\begin{proof}
If $NSA'$ were r.e. then we could use the following algorithm on input string $x$:
\begin{enumerate}
\item Run the algorithm that semidecides $NSA'$. If this halts then go to the next step.
\item \begin{itemize}
\item Run the algorithm that decides if $x$ is the code of a Turing machine. 
\item If $x$ is the code of a Turing machine then it must not be self-accepting, as it's in $NSA'$. 
\item This means it's in $NSA$, so we should halt. 
\item If $x$ is not the code of a Turing machine then we should instead go into an infinite loop. 
\end{itemize}  
\end{enumerate}
This algorithm semidecides $NSA$, which would contradict the fact that $NSA$ is not r.e. 
\end{proof}
\end{frame}

\begin{frame}
\frametitle{Another Argument for $SA$}
We can also use previously proved results to show that $SA$ is not recursive without referencing the halting problem.
\vspace{0.4cm}
\begin{theorem}
$SA$ is not recursive.
\end{theorem}
\begin{proof}
\vspace{0.4cm}
If $SA$ is recursive then $\overline{SA}=NSA'$ is recursive by theorem \ref{T:rec}. This would contradict corollary \ref{C:NSA'}.
\end{proof}
\end{frame}

\begin{frame}
\frametitle{Back to the Halting Problem}
\begin{itemize}
\item We have found two conceptually straightforward languages, one of which is recursive but not r.e. ($SA$), and the other which is not even r.e. ($NSA$). 
\vspace{0.1cm}
\item The arguments are perhaps easier to understand than the somewhat complicated definition of $H''$ used in the proof that the Halting problem is not recursive. 
\vspace{0.1cm}
\item In particular the proof that $NSA$ is not r.e. from theorem \ref{T:NSAalt} uses self-reference to get a contradiction in a much more direct way than the proof that the Halting problem is undecidable does. 
\vspace{0.1cm}
\item However, the Halting problem is independently interesting, and historically important, which is why we discuss it in some detail.
\end{itemize}
\end{frame}

\begin{frame}
\frametitle{The Empty Tape Halting Problem}
\begin{itemize}
\item We know that the halting problem is undecidable. 
\vspace{0.2cm}
\item That is, there is no Turing machine $H$ that can act on $\co(T,I)$ for another Turing machine $T$ and accept if $T(I)$ halts and reject if it does not.
\vspace{0.2cm} 
\item Consider the following decision problem.
\end{itemize}
\begin{definition}[empty tape halting problem]
Is there a Turing machine $E$ that given $\co(T)$ for another Turing machine $T$ will accept if $T$ halts on the empty string and will reject otherwise?
\end{definition}
\end{frame}

\begin{frame}
\frametitle{Solving the Empty Tape Halting Problem}
\begin{itemize}
\item Suppose this machine $E$ exists. 
\item Consider a Turing machine $T$ and an input $I$. 
\item Using $T$ and $I$ we design a new Turing machine $T_I$ that does the following on input $J$.
\begin{enumerate}
\item $T_I$ erases input $J$ from the tape.
\item $T_I$ simulates $T$ on $I$.
\end{enumerate}

\item What happens if we run $E$ on $\co(T_I)$?
\end{itemize}
\vspace{0.2cm}
\scriptsize
\[\text{$E(\co(T_I))$ accepts $\iff T_I$ halts on empty input $\iff T(I)$ halts.}\]
\[\text{$E(\co(T_I))$ rejects $\iff T_I$ does not halt on empty input $\iff T(I)$ does not halt.}\]
\normalsize
\begin{itemize}
\item This looks a lot like solving the halting problem, which we know is impossible. 
\item This can't happen so we will conclude that $E$ should not exist.
\end{itemize}
\end{frame}

\begin{frame}
\frametitle{Formal Version}
\begin{itemize}
\item We want to formalize the argument we just made to show $E$ can't exist.
\item First, assume $E$ solving the ETHP exists.
\item Construct a TM $H$ as follows:
\end{itemize}
\begin{enumerate}
\item Given input $\co(T,I)$ the first thing $H$ does is construct $\co(T_I)$. This is complicated but it can be done.
\item $H$ then simulates $E(\co(T_I))$.
\item  $H(\co(T,I))$ accepts $\iff E(\co(T_I))$ accepts $\iff T(I)$ halts, and rejects otherwise. 
\end{enumerate}
\begin{itemize}
\item $H$ is a well defined Turing machine, and $H$ solves the halting problem. 
\item The only questionable step in the construction of $H$ is the assumption that $E$ exists. 
\item Since $H$ cannot exist we conclude that $E$ cannot exist either, and so the empty tape halting problem is also undecidable.
\end{itemize} 
\end{frame}

\begin{frame}
\frametitle{Reduction}
This is an example of a general technique called \emph{reduction}. The general strategy is as follows.

\begin{enumerate}
\item Start with a decision problem $D$ whose decidability is not known, and a decision problem $U$ that is known to be undecidable (e.g. the halting problem).
\item Show that instances $I_U$ of $U$ can be converted by an algorithm to instances $I_D$ of $D$ so that
\begin{itemize}
\item $I_D$ is a yes instance of $D\iff I_U$ is a yes instance of $U$.
\item $I_D$ is a no instance of $D\iff I_U$ is a no instance of $U$.
\end{itemize}
\item If there is an algorithm for solving $D$ then we could combine this with our instance converting algorithm to find an algorithm for solving $U$.
\item Since $U$ is undecidable no algorithm for solving $U$ exists, and we conclude that an algorithm for solving $D$ cannot exist either.
\end{enumerate}
\end{frame}

\begin{frame}
\frametitle{Ordering by Hardness}

\begin{definition}[reducible]
\emph{$A$ is reducible to $B$} if there is an algorithm that converts instances $I_A$ of $A$ to instances $I_B$ of $B$ so that $I_B$ is a yes instance of $B$ if and only if $I_A$ is a yes instance of $A$, and $I_B$ is a no instance of $B$ if and only if $I_A$ is a no instance of $A$.\end{definition}

\begin{definition}[$A\leq B$]
Given decision problems $A$ and $B$ we write $A\leq B$ if $A$ is reducible to $B$. 
\end{definition} 
\begin{itemize}
\item Informally we think of this as meaning that $B$ is at least as hard as $A$. 
\item I.e. if we had a solution for $B$ we could use it to solve $A$, but we wouldn't necessarily be able to use a solution for $A$ to solve $B$. 
\item With $U$ and $D$ from the previous slide we would write $U\leq D$.
\end{itemize}
\end{frame}

\begin{frame}
\frametitle{A Hierarchy of Decision Problems}
\begin{itemize}
\item This produces a kind of hierarchy of problems (and formal languages), ordered by their relative decidability. 
\item It turns out that all the decidable problems are grouped together at the bottom of this hierarchy. 
\end{itemize}
\begin{theorem}
Let $D$ be a decidable problem, and let $A$ be any other decision problem with at least one yes instance and at least one no instance. Then $D\leq A$.
\end{theorem}
\begin{proof}
\begin{itemize}
\item Since $D$ is decidable, given an instance $I_D$ of $D$ we can find out whether it is `yes' or `no' by running the decision algorithm for $D$. 
\item If $I_D$ is a `yes' we convert it to a `yes' of $A$, and if not we convert it to a `no' of $A$. 
\item So $D\leq A$.
\end{itemize}
\end{proof}
\end{frame}

\begin{frame}
\frametitle{Basic Properties of $\leq$}
\begin{itemize}
\item The $\leq$ relation for decision problems is reflexive (i.e. $A\leq A$), and transitive (i.e. $A\leq B$ and $B\leq C\implies A\leq C$).
\vspace{0.4cm}
\item Why?
\vspace{0.4cm}
\item  $\leq$ is not symmetric.
\vspace{0.4cm}
\item Why?
\end{itemize}
\end{frame}

\begin{frame}
\frametitle{Halts for all Inputs ($HAI$)}
\begin{itemize}
\item Suppose we have the code of a Turing machine $T$. 
\vspace{0.4cm}
\item Is there an algorithm that we can run on $\co(T)$ that tells us whether or not $T(I)$ halts for all inputs $I$? 
\vspace{0.4cm}
\item The answer is no, and one of the exercises for this section is to prove this by reducing the Halting problem to this problem.
\vspace{0.4cm}
\item I.e. to show that $HP\leq HAI$.
\end{itemize} 
\end{frame}

\begin{frame}
\frametitle{Equivalence of Turing machines ($ETM$)}
\begin{itemize}
\item Suppose we have the code for two Turing machines. 
\vspace{0.4cm}
\item Is there an algorithm we can use that will tell us if they do the same thing for all inputs (i.e. if $T_1(I)=T_2(I)$ for all $I$)? 
\vspace{0.4cm}
\item It turns out the answer is no, and we can prove this using the reduction technique.
\vspace{0.4cm}
\item We will show that $HAI\leq ETM$.
\end{itemize} 
\end{frame}

\begin{frame}
\frametitle{$HAI\leq ETM$}
\begin{itemize}
\item An instance of $HAI$ is just a Turing machine $T$, and $T$ is a yes instance if and only if $T(I)$ halts for all $I$. 
\item We want to use $T$ to construct $T_1$ and $T_2$ so that $T(I)$ halts for all $I$ if and only if $T_1(I)=T_2(I)$ for all $I$. 
\item We construct $T_1$ so that it operates as follows:
\begin{enumerate}
\item $T_1(I)$ simulates $T(I)$.
\item If $T(I)$ halts then $T_1(I)$ erases the tape then writes a single `1' before halting. 
\end{enumerate} 
\item We construct $T_2$ so that $T_2(I)$ just writes a single `1' on the tape before halting for all $I$.
\item $T$ is a yes instance of $HAI$ if and only if $T(I)$ halts for all $I$. 
\item This happens if and only if $T_1(I)$ writes a single `1' for all $I$. 
\item But $T_1(I)=1$ for all $I$ if and only if $T_1(I)=T_2(I)$ for all $I$. 
\item So $T$ is a `yes' instance of $HAI$ if and only if $(T_1,T_2)$ is a yes instance of $ETM$. 
\item So $HAI\leq ETM$ and so $ETM$ is undecidable.
\end{itemize} 
\end{frame}



\end{document}