\documentclass[handout]{beamer} 
\title{ITCS 532:\\ 
8. $\mathbf{NP}$-completeness}
\date{}
\author{Rob Egrot}

\usepackage{amsmath, bbold, bussproofs,graphicx}
\usepackage{mathrsfs}
\usepackage{amsthm}
\usepackage{amssymb}
\usepackage[all]{xy}
\usepackage{multirow}
\usepackage{tikz-cd}

\newtheorem{exercise}[theorem]{Exercise}{\bfseries}{\upshape}


\newtheorem{proposition}[theorem]{Proposition}
\newcommand{\bN}{\mathbb{N}}
\newcommand{\bZ}{\mathbb{Z}}
\newcommand{\bQ}{\mathbb{Q}}
\newcommand{\bR}{\mathbb{R}}
\newcommand{\bP}{\mathbb{P}}
\newcommand{\tvs}{\textvisiblespace}
\newcommand{\ra}{\rightarrow}
\newcommand{\la}{\leftarrow}
\newcommand{\co}{\mathbf{code}}

\newcommand{\Po}{\mathbf{P}}
\newcommand{\NP}{\mathbf{NP}}
\newcommand{\SAT}{\mathsf{PSAT}}

\addtobeamertemplate{navigation symbols}{}{%
    \usebeamerfont{footline}%
    \usebeamercolor[fg]{footline}%
    \hspace{1em}%
    \insertframenumber/\inserttotalframenumber
}
\setbeamertemplate{theorems}[numbered]
\begin{document}

\begin{frame}
\titlepage
\end{frame}


\begin{frame}
\frametitle{Verifiers}
\begin{definition}[Verifier]
Given a language $L$ over a finite alphabet $\Sigma$, a verifier for $L$ is a modified Turing machine $V$, where $V$ takes two inputs instead of one (consider the inputs to be separated by a special symbol), and such that
\[x\in L\iff \text{ there is } y\in\Sigma^* \text{ such that } V(x,y) \text{ accepts}\] 
\end{definition}
\begin{itemize}
\item Here $y$ is called a \emph{certificate} for $x$. 
\item A verifier must halt for all inputs. 
\item We say a verifier for $L$ runs in $p$-time if it runs in $p$-time with respect to the length of $x$ for all inputs $(x,y)$. 
\item The certificate is like a proposed solution. 
\item The verifier accepts $x$, which encodes an instance of a decision problem, if and only if there's some proposed solution $y$ that it can check is actually a solution. 
\end{itemize}
\end{frame}

\begin{frame}
\frametitle{Example}
\begin{example}[Hamiltonian circuits]
\begin{itemize}
\item An instance of the Hamiltonian circuit problem is a graph $G$. 
\item A certificate for $G$ is just a sequence $s$ of vertices (in coded form). 
\item A verifier $V$ for the Hamiltonian  circuit problem would accept on input $(\co(G),\co(s))$ if and only if $s$ defines a Hamiltonian circuit in $G$.
\end{itemize} 
\end{example}
\vspace{0.5cm}
\begin{theorem}
A language has a verifier if and only if it is recursively enumerable.
\end{theorem}
\end{frame}


\begin{frame}
\frametitle{The Class $\NP$}
\begin{definition}
$\NP$ is the class of all decision problems that have a $p$-time verifier.
\end{definition}
\vspace{1cm}
\begin{itemize}   
\item Note that $\NP$ does \emph{not} stand for \emph{non-polynomial}. 
\vspace{1cm}
\item $\NP$ actually stands for \emph{non-deterministic polynomial}. 
\vspace{1cm}
\item The name comes from an alternative definition of the class $\NP$, which we will see now.
\end{itemize} 
\end{frame}


\begin{frame}
\frametitle{Non-deterministic Turing machines}
\begin{itemize}
\item A non-deterministic Turing machine ($NTM$) is a Turing machine variant whose transition function $\delta$ has the form:
\[\delta:Q\times\Sigma\cup\{:,\tvs\}\to\wp\Big(Q\times \big(\Sigma\cup\{\tvs,\la,\ra\}\big)\Big)\]
\item That is, $\delta$ is now a function that takes as input the state of the machine and the symbol being read by the tape head, and returns a \emph{set} of pairs of new states and tape head actions. 
\item Obviously it doesn't make sense for a machine to enter a set of states, or write a set of symbols on the tape.
\item We interpret the output of $\delta$ as a \emph{choice}. 
\item In a run of an NTM, at every step the machine picks one pair of states and tape head actions from the set of possibilities defined by $\delta$. 
\item This is why these machines are called non-deterministic.
\end{itemize} 
\end{frame}

\begin{frame}
\frametitle{Non-deterministic Turing machines}
\begin{itemize}
\item For example, suppose the NTM is in state $q$ and reading symbol $\sigma$, and that $\delta(q,\sigma)=\{(q_1,\sigma_1),(q_2,\sigma_2)\}$. 
\item Then the machine can either go into state $q_1$ and write $\sigma_1$ on the tape, or go into state $q_2$ and write $\sigma_2$ on the tape.  

\item Given an input $I$ and a non-deterministic Turing machine $N$, we say $N(I)$ \emph{accepts} if there is some sequence of $\delta$ choices resulting in the accept state. 
\item We say $N(I)$ \emph{rejects} if every possible sequence of $\delta$ choices results in rejection. 
\item If neither of these is true then $N(I)$ is undefined. 
\item A sequence of $\delta$ choices for $N(I)$ that either reaches a halt state or continues forever is called a \emph{run} of $N(I)$. 
\item If every possible run of $N(I)$ halts for all possible inputs $I$ then we say $N$ is a \emph{decider}.
\end{itemize} 
\end{frame}

\begin{frame}
\frametitle{The Run Time of an NTM}
\begin{itemize}
\item It's not obvious how the run time of an NTM should be defined in general, but for deciders we do so as follows:
\begin{itemize} 
\item For an input $I$ the run time of $N(I)$ is defined to be the length of the longest possible run of $N(I)$. 
\item We define the run time of $N$ as a function $f:\mathbb{N}\to \mathbb{N}$ by 
\[f(n)=\max\{ \text{run time of }N(I) \text{ such that length }I \text{ is }n\}\]
\item As usual we use big $O$ notation so we can talk about $f$ without having to explicitly define it.
\end{itemize}
\item Note that in this definition of run time we use the longest run, even if there's a shorter run that accepts. 
\item For example, if for some $N$ and some $I$ there are exactly two runs, one which accepts after 2 steps and one which rejects after 10 steps, then the run time of $N(I)$ is 10. 
\end{itemize}
\end{frame}

\begin{frame}
\frametitle{Determinism vs Non-Determinism}
The following theorem says NTMs are not more powerful than TMs (if we don't care about run time).

\begin{theorem}
Let $L$ be a formal language over a finite alphabet. Then $D_L$ (the decision problem associated with $L$) is solvable by a Turing machine if and only if it is solvable by a non-deterministic Turing machine. 
\end{theorem}
\begin{proof}
\begin{itemize}
\item Clearly every ordinary (i.e. deterministic) Turing machine $T$ is equivalent to a non-deterministic Turing machine $T'$.
\item  Just define $\delta'$ by e.g. setting $\delta'(q,\sigma)=\{(q',\sigma')\}$ when $\delta(q,\sigma)=(q',\sigma')$.
\item The converse is more difficult, but the basic idea is that we can use dovetailing to compute every possible run of an $NTM$ using a specially designed standard $TM$.
\end{itemize}
\end{proof}
\end{frame}

\begin{frame}
\frametitle{NTMs and Verifiers}
\begin{theorem}\label{T:NP}
Let $L$ be a formal language over a finite alphabet. Then $L$ has a $p$-time verifier if and only if $L$ can be decided by a non-deterministic Turing machine in $p$-time.
\end{theorem}
\vspace{0.5cm}
\textbf{Proof}
\vspace{0.5cm}
\begin{itemize}
\item If $L$ is a language, we must show that:
\vspace{0.5cm}
\begin{enumerate}
\item If $L$ has a $p$-time verifier $V$, then we can construct an NTM $N_V$ that decides $L$ in $p$-time.
\vspace{0.5cm}
\item If there is an NTM $N$ that decides $L$ in $p$-time, then we can construct a $p$-time verifier $V_N$ for $L$.
\end{enumerate}
\end{itemize}
\end{frame}

\begin{frame}
\frametitle{Has Verifier Implies Decided by NTM}
\begin{itemize}
\item Let $V$ be a $p$-time verifier for $L$. 
\item $V$ runs in $p$-time, so there is $n\in\bN$ such that $V(x,y)$ must halt within $C|x|^k$ steps whenever $|x|\geq n$. 
\item The idea is that we define a non-deterministic Turing machine $N_V$ that, given $x$, first constructs a string $y$ and then runs $V(x,y)$. 
\item Implementing this idea is a little tricky, as there are some details we need to be careful with. 
\item First, note that there are a finite number of strings whose length is less than $n$. 
\item For each $x\in L$ with $|x|<n$, define $y_x$ to be a string of minimal length such that $V(x,y_x)$ accepts. 
\item Since there are a finite number of these $y_x$ elements, define $l$ to be the length of the longest of these. 
\end{itemize}
\end{frame}

\begin{frame}
\frametitle{Has Verifier Implies Decided by NTM}
\begin{itemize}
\item We define $N_V$ so that a `run' of $N_V(x)$ is as follows:
\begin{enumerate}
\item Check the length of $x$.
\item If $|x|\geq n$ then non-deterministically construct $y$ so that $|y|\leq C|x|^k$. Otherwise non-deterministically construct $y$ such that $|y|\leq l$.
\item Run $V(x,y)$ (this part is deterministic).
\end{enumerate}
\item Consider first the case where $|x|\geq n$. 
\item If $x\in L$ then there is a certificate $y$ such that $V(x,y)$ accepts within $C|x|^k$ steps. 
\item The upper bound on the number of steps means that $|y|\leq C|x|^k$ too. 
\item This means there is an accepting run for $N_V(x)$. 
\item Alternatively, if $x\notin L$ then $V(x,y)$ rejects for all $y$, and this is obviously still true if we restrict to $y$ where $|y|\leq C|x|^k$. 
\item Constructing $y$ takes $p$-time, and running $V(x,y)$ takes $p$-time. 
\item So the longest possible run of $N_V(x)$ is also $p$-time.
\end{itemize}
\end{frame}

\begin{frame}
\frametitle{Has Verifier Implies Decided by NTM}
\begin{itemize}
\item Consider now the case where $|x|<n$. 
\item Suppose $x\in L$. 
\item Then there is $y_x$ with $|y_x|\leq l$ and such that $V(x,y_x)$ accepts, so $N_V(x)$ accepts. 
\item Alternatively, if $x\notin L$ then there is no $y$ such that $V(x,y)$ accepts, so $N_V(x)$ rejects. 
\item Combining this with the case where $|x|\geq n$ we see that $N_V$ decides $L$. 
\item What is the running time of $N_V$?  
\item We don't care about the running time of $N_V(x)$ in the case where $|x|< n$. 
\item We have also shown that, for all $x$ with $|x|\geq n$, a run of $N_V(x)$ is polynomially bounded. 
\item Thus $N_V$ runs in $p$-time as required.
\end{itemize}
\end{frame}

\begin{frame}
\frametitle{Decided by NTM Implies Has Verifier}
\begin{itemize}
\item Given $N$ that decides $L$ in $p$-time we define a verifier $V_N$ where a string $y$ that is a potential certificate encodes a sequence $n_0,n_1,\ldots,n_k$ of natural numbers. 
\vspace{0.5cm}
\item Every time $N$ `makes a choice', we can assign an ordering of the possible alternatives. 
\vspace{0.5cm}
\item For example, if \[\delta:(q,\sigma)\mapsto \{(q_0,\sigma_0), (q_1,\sigma_1), (q_2,\sigma_2)\}\] 
we can arbitrarily choose an order and assume without loss of generality that $\{(q_0,\sigma_0), (q_1,\sigma_1), (q_2,\sigma_2)\}$ is the ordered sequence $((q_0,\sigma_0), (q_1,\sigma_1), (q_2,\sigma_2))$. 
\end{itemize}
\end{frame}

\begin{frame}
\frametitle{Decided by NTM Implies Has Verifier}
\begin{itemize}
\item We define $V_N$ so that $V_N(x,y)$ operates by computing $N(x)$, and such that the `choices' made by $N$ correspond to the numbers in the sequence $n_0,n_1,\ldots,n_k$ defined by $y$. 
\vspace{0.5cm}
\item If $y$ does not correspond to such a sequence, or if this sequence is incompatible with $N(x)$ in any way, then we define $V_N$ so that $V_N(x,y)$ rejects. 
\vspace{0.5cm}
\item If there is an accepting run of $N(x)$ then this will correspond to a sequence of `correct' choices, and thus to a $y$ such that $V_N(x,y)$ accepts. 
\vspace{0.5cm}
\item Alternatively, if there is no accepting run then $V_N(x,y)$ will reject for all $y$.
\end{itemize}
\end{frame}

\begin{frame}
\frametitle{Decided by NTM Implies Has Verifier}
\begin{itemize}
\item As $N$ runs in $p$-time there are $C,m,k\in \bN$ such that $N(x)$ takes at most $C|x|^k$ steps whenever $|x|\geq m$. 
\vspace{0.5cm}
\item Thus $N$ makes at most $C|x|^k$ choices when computing $N(x)$. 
\vspace{0.5cm}
\item Assuming we have picked a sensible encoding scheme, it will be possible to look up the numbers of the choices appropriately in $p$-time and act accordingly. 
\vspace{0.5cm}
\item Thus $V_N$ runs in $p$-time as required.
\end{itemize}
\end{frame}

\begin{frame}
\frametitle{Revisiting $\NP$}
\begin{corollary}
$\NP$ is the class of all decision problems that can be solved by a non-deterministic Turing machine in $p$-time.
\end{corollary}
\begin{proof}
\begin{itemize}

\item Remember that we defined $\NP$ to be the class of decision problems for which a $p$-time verifier exists. 
\item By theorem \ref{T:NP} this is exactly the class of problems solvable in $p$-time by an NTM.
\end{itemize}
\end{proof} 
\end{frame}

\begin{frame}
\frametitle{$\Po$ vs $\NP$}
\begin{itemize}
\item Question: Why is it obvious that $\Po\subseteq \NP$?
\item Is it possible that $\Po=\NP$? 
\item This is possibly the most important open question in theoretical computer science. 
\item The Clay Mathematics Institute lists it as one of their seven `Millennium Problems'. 
\item The $\Po$ vs. $\NP$ question symbolizes something of profound practical significance: 
\item Is it intrinsically harder to find solutions than to verify them? 
\item If the answer to this is `no', i.e. if $\Po=\NP$, then there could be efficient algorithms for solving all kinds of currently hard problems (including those used in encryption systems).
\item Most computer scientists believe that $\Po\neq\NP$. 
\item This has turned out to be very hard to prove. 
\item But there are lots of $\NP$ problems that nobody has ever found a $p$-time algorithm for.  
\end{itemize} 
\end{frame}

\begin{frame}
\frametitle{$\NP$-hardness}
\begin{definition}[$\NP$-hard]
A decision problem $B$ is $\NP$-hard if for all decision problems $A\in \NP$ we have $A\leq_p B$. 
\end{definition}
\begin{itemize}
\item Consider the definition above and think about what it means. 
\item A problem is $\NP$-hard if every problem in $\NP$ reduces to it in polynomial time. 
\item Informally, we interpret this as meaning an $\NP$-hard problem is at least as hard as the hardest problems in $\NP$. 
\item If we could find a $p$-time algorithm for solving an $\NP$-hard problem then we would be able to find $p$-time algorithms for all $\NP$ problems. 
\item In other words it would show that $\Po=\NP$.  
\end{itemize} 
\end{frame}

\begin{frame}
\frametitle{$\NP$-completeness}
\begin{itemize}
\item It's not immediately obvious that $\NP$-hard problems exist.
\vspace{0.5cm}
\item But, as we will see shortly, they do, and, moreover, there are even $\NP$-hard problems that are also members of $\NP$. 
\vspace{0.5cm}
\item This leads us to the following definition.  
\end{itemize} 
\vspace{0.5cm}
\begin{definition}[$\NP$-complete]
A decision problem is $\NP$-complete if it is $\NP$-hard and also a member of $\NP$.
\end{definition} 

\end{frame}

\begin{frame}
\frametitle{$\NP$-completeness}
\begin{itemize}
\item We will give an example of an $\NP$-complete problem soon. 
\item Once we have one $\NP$-complete problem, however, it becomes much easier to find more, due to the following theorem.  
\end{itemize} 
\begin{theorem}\label{T:ctrans}
If $B$ is $\NP$-complete and $C\in\NP$, then $B\leq_p C\implies C$ is $\NP$-complete.
\end{theorem}
\begin{proof}
\begin{itemize}
\item Let $A\in \NP$. 
\item We need to show that $A\leq_p C$. 
\item But we know $A\leq_p B$ as $B$ is $\NP$-complete, and $B\leq_p C$ by assumption, so this follows from transitivity of $\leq_p$.
\end{itemize}
\end{proof}
\end{frame}

\begin{frame}
\frametitle{$\SAT$}
\begin{itemize}
\item To define our first $\NP$-complete problem we need some preliminary definitions from propositional logic.
\end{itemize} 
\begin{definition}[Boolean formula]
A Boolean formula is a finite string of propositional variables, brackets, and logical symbols from $\{\vee,\wedge,\neg\}$ constructed according to the following rules:
\begin{enumerate}
\item $p$ is a Boolean formula for all propositional variables $p$.
\item If $\phi$ and $\psi$ are Boolean formulas then $(\phi\vee\psi)$ and $(\phi\wedge\psi)$ are Boolean formulas.
\item If $\phi$ is a Boolean formula then $\neg\phi$ is a Boolean formula.
\end{enumerate} 
\end{definition}
\end{frame}

\begin{frame}
\frametitle{$\SAT$}

\begin{definition}[Satisfiable]
A Boolean formula $\phi$ is satisfiable if there is an assignment of 1 or 0 (`true' or `false') to every propositional variable appearing in $\phi$ such that $\phi$ is `true' in the resulting truth table.
\end{definition}
\vspace{1cm}
\begin{definition}[$\SAT$]
$\SAT$ is the decision problem that asks if a given Boolean formula is satisfiable.
\end{definition}

\end{frame}

\begin{frame}
\frametitle{$\SAT$ is $\NP$-complete}

\begin{theorem}
$\SAT$ is $\NP$-complete.
\end{theorem}
\textbf{Proof sketch}
\begin{itemize}
\item We will sketch a proof without going too far into the details. 
\item First we must check that $\SAT$ is in $\NP$.
\item By definition, a problem is in $\NP$ if it has a $p$-time verifier. 
\item We can check if a given Boolean formula evaluates to `true' when given a specific variable assignment in $p$-time. 
\item To do this we can rewrite a Boolean formula into something called \emph{reverse Polish notation} using the \emph{shunting yard algorithm}. 
\item In reverse Polish notation we can evaluate expressions just by reading them from left to right, which takes linear time. 
\item Since the shunting yard algorithm runs in linear time too, we can evaluate Boolean formulas for specific values of their propositions in linear time.
\end{itemize} 
\end{frame}

\begin{frame}
\frametitle{$\SAT$ is $\NP$-complete}
\begin{itemize}
\item Now we must show every problem in $\NP$ reduces to $\SAT$.
\vspace{0.2cm}
\item Let $A$ be a decision problem in $\NP$. 
\vspace{0.2cm}
\item We know there's an NTM that decides $A$ in $p$-time. 
\vspace{0.2cm}
\item We must find a $p$-time algorithm that takes an instance $I$ of $A$ and turns it into a Boolean formula that is satisfiable if and only if $I$ is a `yes' instance. 
\vspace{0.2cm}
\item Since all we know about $A$ is that it is in $\NP$ we're going to have to use the NTM that decides $A$ in our reduction. 
\vspace{0.2cm}
\item Let $N$ be an NTM that decides $A$. 
\vspace{0.2cm}
\item The idea is that we will create a Boolean formula that expresses the idea that $N(I)$ has an accepting run. 
\vspace{0.2cm}
\item This formula will be satisfiable if and only if an accepting run exists. I.e. if and only if $I$ is a `yes' instance.
\end{itemize} 
\end{frame}

\begin{frame}
\frametitle{$\SAT$ is $\NP$-complete}
\begin{itemize}
\item This idea is quite similar to how we showed the entscheidungsproblem is undecidable. 
\vspace{0.3cm}
\item This time we don't have the expressive power of first-order logic to work with. 
\vspace{0.3cm}
\item For the entscheidungsproblem proof we used predicates to describe the state of the tape at each step in the computation. 
\vspace{0.3cm}
\item In propositional logic we don't have predicates, so we have to use propositional variables for the same purpose. 
\vspace{0.3cm}
\item But our formula can only use a finite number of propositional variables.
\vspace{0.3cm} 
\item This is where the bounded running time of $N$ helps us.
\end{itemize} 
\end{frame}

\begin{frame}
\frametitle{$\SAT$ is $\NP$-complete}
\begin{itemize}
\item As in the entscheidungsproblem proof, we think of a run of a TM as a two dimensional grid. 
\vspace{0.3cm}
\item The rows of the grid represent the tape at successive steps of the calculation. 
\vspace{0.3cm}
\item Since $N$ is non-deterministic, $N(I)$ is associated with multiple different runs, and so with multiple different grids. 
\vspace{0.3cm}
\item Since $N$ runs in $p$-time, we can suppose that for large $n$ its run time on an input of length $n$ is less than $Cn^k$. 
\vspace{0.3cm}
\item Suppose the length of $I$ is $n$. Then this means that every grid associated with a run of $N(I)$ must fit into $Cn^k\times Cn^k$. 
\vspace{0.3cm}
\item Cells outside of this range are inaccessible to a machine halting in fewer than $Cn^k$ steps, so we can ignore them.
\end{itemize} 
\end{frame}

\begin{frame}
\frametitle{$\SAT$ is $\NP$-complete}
\begin{itemize}
\item Let $Q$ be the set of states of $N$, and let $\Sigma$ be its (finite) set of symbols. 
\vspace{0.3cm}
\item Our Boolean formula $\phi$ will use the following set of propositional variables:  
\vspace{0.3cm}
\begin{itemize}
\item For every $i,j\leq Cn^k$ and every $\sigma\in \Sigma\cup\{:,\tvs\}$ we have $p_{i,j,\sigma}$. This variable is supposed to be `true' if $\sigma$ is written on the tape at position $i$ and time $j$.  
\vspace{0.3cm}
\item For every $i,j\leq Cn^k$ and every $q\in Q$ we have $p_{i,j,q}$. This variable is supposed to be `true' if the machine is in state $q$ and the tape head is at position $i$ at time $j$.
\end{itemize}
\end{itemize} 
\end{frame}

\begin{frame}
\frametitle{$\SAT$ is $\NP$-complete}
\begin{itemize}
\item The idea is that $\phi$ will express the following:
\end{itemize} 
\begin{enumerate}
\item Every cell in the grid must contain exactly one symbol. 
\item The first row of the grid corresponds to the input $I$.
\item The tape head starts in the right place.
\item The machine is in exactly one state at every step.
\item The machine starts in the start state.
\item The tape head is always somewhere on the tape. 
\item Every row except the first must represent a configuration of $N(I)$ that follows from the configuration corresponding to the previous row according to some choice defined by the transition function $\delta$ of $N$.
\item At some point the machine enters the accept state.
\end{enumerate}
\end{frame}

\begin{frame}
\frametitle{$\SAT$ is $\NP$-complete}
\begin{itemize}
\item It helps to break down the construction of $\phi$ into parts. 
\vspace{0.5cm}
\item For example, we can express 1 using
\[\phi_1=\bigwedge_{0\leq i,j\leq Cn^k}\Big((\bigvee_{\sigma\in\Sigma\cup\{:,\tvs\}}p_{i,j,\sigma})\wedge \bigwedge_{\sigma\neq\sigma'\in\Sigma\cup\{:,\tvs\}}\neg (p_{i,j,\sigma}\wedge p_{i,j,\sigma'})\Big)\]
\item The construction of $\phi$ turns out to take $O(n^{2k})$ steps. 
\vspace{0.5cm}
\item  $\phi$ will be satisfiable if and only if $N(I)$ has an accepting run.
\vspace{0.5cm}
\item  I.e. a sequence of $\delta$ choices leading to an accept state. 
\end{itemize} 
\end{frame}

\begin{frame}
\frametitle{$\SAT$ is $\NP$-complete}
\begin{itemize}
\item What about `short' inputs, such that the run time bound $Cn^k$ does not apply? 
\vspace{0.5cm}
\item There are only a finite number of these, so there's still an upper bound on the size of the `grids' representing the computations. 
\vspace{0.5cm}
\item The construction time of $\phi$ for these short inputs doesn't matter because the definition of $p$-time only requires polynomial bounding for `large' input lengths.
\vspace{0.5cm}
\item So we have shown that $A\leq_p\SAT$.
\end{itemize} 
\end{frame}

\begin{frame}
\frametitle{Some Final Comments}
\begin{itemize}
\item Using $p$-time reduction we can use $\SAT$ to show that other problems are $\NP$-complete too. 
\vspace{0.2cm}
\item It turns out that lots of interesting combinatorial problems are.
\vspace{0.2cm} 
\item Important examples include the Hamiltonian circuit and Hamiltonian path problems, the traveling salesman decision problem, the knapsack problem, and many more.
\vspace{0.2cm}
\item Finally, there are problems that are $\NP$-hard but not in $\NP$ (and so not $\NP$-complete). 
\vspace{0.2cm}
\item The halting problem is one example. 
\vspace{0.2cm}
\item It is obviously not in $\NP$ as it isn't even decidable, but we can show that $\SAT$ reduces to it in $p$-time. 
\vspace{0.2cm}
\item So by theorem \ref{T:ctrans} the halting problem is $\NP$-hard.  
\end{itemize} 
\end{frame}

\end{document}