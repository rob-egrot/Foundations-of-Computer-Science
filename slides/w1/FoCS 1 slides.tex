\documentclass[handout]{beamer} 
\title{ITCS 532:\\ 1. Models of Computation}
\date{}
\author{Rob Egrot}

\usepackage{amsmath, bbold, bussproofs,graphicx}
\usepackage{mathrsfs}
\usepackage{amsthm}
\usepackage{amssymb}
\usepackage[all]{xy}
\usepackage{multirow}
\usepackage{tikz-cd}


\newtheorem{proposition}[theorem]{Proposition}
\newcommand{\bN}{\mathbb{N}}
\newcommand{\bZ}{\mathbb{Z}}
\newcommand{\bQ}{\mathbb{Q}}
\newcommand{\bR}{\mathbb{R}}
\newcommand{\bP}{\mathbb{P}}
\newcommand{\tvs}{\textvisiblespace}
\newcommand{\ra}{\rightarrow}
\newcommand{\la}{\leftarrow}

\addtobeamertemplate{navigation symbols}{}{%
    \usebeamerfont{footline}%
    \usebeamercolor[fg]{footline}%
    \hspace{1em}%
    \insertframenumber/\inserttotalframenumber
}
\setbeamertemplate{theorems}[numbered]
\begin{document}

\begin{frame}
\titlepage
\end{frame}

\begin{frame}
\frametitle{What is computation?}
\begin{itemize}
\item Given a set $\Gamma$ of first-order sentences, and a first-order sentence $\phi$, does $\Gamma\models \phi$?
\vspace{0.5cm}
\item Is there an algorithm that will always answer the above question correctly?
\vspace{0.5cm}
\item What exactly is an algorithm anyway?
\vspace{0.5cm}
\item What is computation?
\vspace{0.5cm}
\item How does this relate to modern computers?
\end{itemize}

\end{frame}

\begin{frame}
\frametitle{Finite State Machines}
\begin{itemize}
\item A \emph{finite state machine} (FSM), AKA \emph{finite automaton}, is an abstract machine.
\vspace{0.5cm}
\item At any moment an FSM is in one of a finite number of possible states.
\vspace{0.5cm}
\item The state changes in response to input.
\vspace{0.5cm}
\item There are only a finite number of possible inputs.
\item E.g:
\begin{itemize}
\item ticket machines,
\item vending machines,
\item etc.
\end{itemize}
\end{itemize}

\end{frame}

\begin{frame}
\frametitle{FSM - formal definition}
An FSM consists of:
\begin{enumerate}
\item A finite set of states $Q$.
\item A distinguished set $H\subseteq Q$. This set contains the \emph{halting} states of $M$. If the machine gets to a state in $H$ then it stops running. 
\item A special starting state $q_0\in Q$.
\item A finite set of possible inputs $\Sigma$ sometimes called the \emph{alphabet} of $M$.
\item A transition function $\delta:(Q\setminus H)\times \Sigma\to Q$. This function controls state change. I.e. $\delta(q,\sigma)=r$ means \emph{if in state $q$ go to state $r$ when receiving input $\sigma$}.
\end{enumerate} 
\end{frame}

\begin{frame}
\frametitle{Warning 1}
\begin{itemize}
\item There are other definitions for FSMs.
\vspace{0.5cm}
\item E.g:
 \begin{itemize}
\item $\delta$ may be partial. 
\item In this case there must be a rule for dealing with undefined situations.
\end{itemize}
\vspace{0.5cm}
\item Alternative definitions will be essentially equivalent to ours.
\end{itemize}
\end{frame}

\begin{frame}
\frametitle{Warning 2}
\begin{itemize}
\item In the real world, things like ticket machines take user input in real time.
\vspace{0.2cm}
\item Users might walk away before completing their purchase.
\vspace{0.2cm}
\item The design of the machine must deal with this.
\vspace{0.2cm}
\item E.g. after some time of inactivity state of machine must reset.
\vspace{0.2cm}
\item Waiting time should not be too short or too long.
\vspace{0.2cm}
\item Not our problem.
\vspace{0.2cm}
\item We abstract away time, and treat input as a pre-determined finite sequence of events.
\vspace{0.2cm}
\item We don't worry about what happens after that.
\end{itemize}
\end{frame}

\begin{frame}
\frametitle{Running an FSM}
\begin{itemize}
\item The possible inputs are represented by the symbols in $\Sigma$.
\vspace{0.2cm}
\item Sequences of inputs correspond to finite strings from $\Sigma$.
\vspace{0.2cm}
\item $\Sigma^*$ is set of all finite strings from $\Sigma$.
\vspace{0.2cm}
\item We think of the \emph{input} of an FSM to be a string from $\Sigma^*$.
\vspace{0.2cm}
\item Machine acts on each symbol in the input in turn, and changes state according to $\delta$.
\vspace{0.2cm}
\item The output is just the state after acting on the final symbol of the input.
\end{itemize}
\end{frame}

\begin{frame}
\frametitle{Representing an FSM}
This diagram represents an FSM that looks for the sequence $abc$ inside strings composed of letters from the alphabet $\Sigma=\{a,b,c\}$.
\begin{center}
\begin{tikzpicture}[scale=0.2]
\tikzstyle{every node}+=[inner sep=0pt]
\draw [black] (16.6,-30.5) circle (3);
\draw (16.6,-30.5) node {$q_0$};
\draw [black] (29.5,-30.5) circle (3);
\draw (29.5,-30.5) node {$A$};
\draw [black] (42.7,-30.5) circle (3);
\draw (42.7,-30.5) node {$B$};
\draw [black] (54.8,-30.5) circle (3);
\draw (54.8,-30.5) node {$C$};
\draw [black] (54.8,-30.5) circle (2.4);
\draw [black] (15.277,-27.82) arc (234:-54:2.25);
\draw (16.6,-23.25) node [above] {$b,c$};
\fill [black] (17.92,-27.82) -- (18.8,-27.47) -- (17.99,-26.88);
\draw [black] (40.024,-31.838) arc (-70.68221:-109.31779:11.861);
\fill [black] (40.02,-31.84) -- (39.1,-31.63) -- (39.43,-32.57);
\draw (36.1,-33.01) node [below] {$b$};
\draw [black] (52.201,-31.974) arc (-69.26268:-110.73732:9.745);
\fill [black] (52.2,-31.97) -- (51.28,-31.79) -- (51.63,-32.72);
\draw (48.75,-33.11) node [below] {$c$};
\draw [black] (32.007,-28.874) arc (114.31851:65.68149:9.938);
\fill [black] (32.01,-28.87) -- (32.94,-29) -- (32.53,-28.09);
\draw (36.1,-27.49) node [above] {$a$};
\draw [black] (18.215,-27.978) arc (141.48882:38.51118:14.614);
\fill [black] (18.21,-27.98) -- (19.1,-27.66) -- (18.32,-27.04);
\draw (29.65,-21.96) node [above] {$b$};
\draw [black] (30.823,-33.18) arc (54:-234:2.25);
\draw (29.5,-37.75) node [below] {$a$};
\fill [black] (28.18,-33.18) -- (27.3,-33.53) -- (28.11,-34.12);
\draw [black] (26.969,-32.089) arc (-66.58879:-113.41121:9.864);
\fill [black] (26.97,-32.09) -- (26.04,-31.95) -- (26.43,-32.87);
\draw (23.05,-33.4) node [below] {$a$};
\draw [black] (19.214,-29.049) arc (111.00052:68.99948:10.702);
\fill [black] (19.21,-29.05) -- (20.14,-29.23) -- (19.78,-28.3);
\draw (23.05,-27.84) node [above] {$c$};
\end{tikzpicture}
\end{center}
If it gets to state $C$ it halts, and we have a success. Otherwise once it reaches end of input we consider it a failure. I.e., input does not contain $abc$.
\end{frame}

\begin{frame}
\frametitle{The formal version}
The machine on the previous slide has this formal description:
\begin{itemize}
\item $Q=\{q_0,A,B,C\}$.
\item $\Sigma=\{a,b,c\}$.
\item $\delta:\begin{cases}
(q_0,a)\mapsto A\\
(q_0,b)\mapsto q_0\\
(q_0,c)\mapsto q_0\\
(A,a)\mapsto A\\
(A,b)\mapsto B\\
(A,c)\mapsto q_0\\
(B,a)\mapsto A\\
(B,b)\mapsto q_0\\
(B,c)\mapsto C\\
 \end{cases}$
\item $q_0$ (the starting state).
\item $H=\{C\}$ (the set of halt states).
\end{itemize}
\end{frame}

\begin{frame}
\frametitle{Limitations of FSMs}
\begin{itemize}
\item Finite state machines are useful, but they are quite limited.
\vspace{0.4cm}
\item E.g: 
\begin{itemize}
\item No FSM to check if a graph is connected.
\item No FSM to check if a given set of polynomial equations has a solution.
\item No FSM to check if a number in binary is prime.
\item Etc.
\end{itemize}
\vspace{0.2cm}
\item A simpler example:
\vspace{0.2cm}
\item Why is there no FSM that can add two binary numbers together?
\end{itemize}
\end{frame}

\begin{frame}
\frametitle{Memory problems}
\begin{itemize}
\item Many of the limitations of FSMs come down to memory.
\vspace{0.2cm}
\item Specifically, \emph{lack} of memory.
\vspace{0.2cm}
\item An FSM records information only in its state.
\vspace{0.2cm}
\item Since it has a finite number of states, but inputs can be arbitrarily long, its ability to distinguish between inputs is limited.
\vspace{0.2cm}
\item Real computers also have finite memory, but the amount increases all the time.
\vspace{0.2cm}
\item Theoretical computer scientists often prefer to study abstract computation by assuming memory is infinite.
\end{itemize}
\end{frame}

\begin{frame}
\frametitle{Turing machines}
\begin{itemize}
\item A Turing machine (TM) is another abstract machine.
\item Turing defined this concept to capture the notion of an algorithm.
\item It is similar to an FSM but it also has an infinite \emph{tape}.
\item The TM can write on the tape, and can access things it has written before.
\item The tape provides the TM with an infinite memory.
\item A TM has a tape head which moves around on the tape reading symbols.
\item During operation, the TM reads a symbol and checks its current state.
\item Then it moves into a new state, and acts on the tape in some way, either by writing a new symbol or moving the tape head. 
\end{itemize}
\end{frame}

\begin{frame}
\frametitle{Turing machines - formal version $(Q,\Sigma,q_0,H,\delta)$}
\begin{enumerate}
\item A finite set of states $Q$.
\item A finite alphabet $\Sigma$. Also special symbols $\tvs$ (for blank spaces) and $:$ (for start of tape). 
\item A one way infinite tape consisting of numbered squares starting at $0$. Each square contains one symbol from $\Sigma\cup\{\tvs,:\}$. Symbol $:$ at square $0$ (only).
\item A tape head that moves up and down the tape reading symbols. 
\item A distinguished starting state $q_0$.
\item A set $H\subseteq Q$ of halting states (maybe partitioned into `accept' states and `reject' states).
\item A function $\delta: (Q\setminus H)\times\Sigma\cup\{\tvs,:\} \to Q \times (\Sigma\cup\{\tvs,\rightarrow,\leftarrow\})$.
\end{enumerate}
\end{frame}

\begin{frame}
\frametitle{Running a Turing machine}
\begin{itemize}
\item The \emph{input} of a TM is the initial state of the tape. 
\item We assume that the starting tape always contains a finite string from $\Sigma^*$ followed by an infinite sequence of $\tvs$ symbols.
\item A \emph{run} of a Turing machine on input $I$ is a sequence of abstract triples representing the state the machine is in, the position of the tape head, and the current state of the tape. 
\item A run is completely determined by the input $I$ on the tape and the function $\delta$. 
\item The \emph{output} can either be defined to be the final halting state (e.g. \emph{accept} or \emph{reject}), or the state of the tape up to the first $\tvs$ after halting.
\item Runs don't always halt!
\end{itemize}
\end{frame}

\begin{frame}
\frametitle{Example - unary addition of 1}
Let $\Sigma=\{1\}$. This machine adds one to a natural number represented in unary notation.
\begin{center}
\begin{tikzpicture}[scale=0.2]
\tikzstyle{every node}+=[inner sep=0pt]
\draw [black] (3.3,-34.6) circle (3);
\draw (3.3,-34.6) node {$q_0$};
\draw [black] (14.5,-34.6) circle (3);
\draw (14.5,-34.6) node {$q_1$};
\draw [black] (25,-34.6) circle (3);
\draw (25,-34.6) node {$H$};
\draw [black] (25,-34.6) circle (2.4);
\draw [black] (6.3,-34.6) -- (11.5,-34.6);
\fill [black] (11.5,-34.6) -- (10.7,-34.1) -- (10.7,-35.1);
\draw (8.9,-35.1) node [below] {$:,\rightarrow$};
\draw [black] (17.5,-34.6) -- (22,-34.6);
\fill [black] (22,-34.6) -- (21.2,-34.1) -- (21.2,-35.1);
\draw (19.75,-35.1) node [below] {$\tvs,1$};
\draw [black] (13.177,-31.92) arc (234:-54:2.25);
\draw (14.5,-27.35) node [above] {$1,\rightarrow$};
\fill [black] (15.82,-31.92) -- (16.7,-31.57) -- (15.89,-30.98);
\end{tikzpicture}
\end{center}
\end{frame}

\begin{frame}
\frametitle{Example - unary addition of two natural numbers}
This machine uses the alphabet $\{1,*\}$ and correct input is of form $*,a*,*b,a*b$ followed by blanks  ($a$ and $b$ are strings containing only $1$). Here $a$ and $b$ represent the numbers to be added together in unary notation. Output is the unary number that is left on the tape when the machine accepts. 
\begin{center}
\begin{tikzpicture}[scale=0.2]
\tikzstyle{every node}+=[inner sep=0pt]
\draw [black] (10.5,-19.2) circle (3);
\draw (10.5,-19.2) node {$q_0$};
\draw [black] (24.1,-19.2) circle (3);
\draw (24.1,-19.2) node {$q_1$};
\draw [black] (36,-19.3) circle (3);
\draw (36,-19.3) node {$q_2$};
\draw [black] (48.4,-19.3) circle (3);
\draw (48.4,-19.3) node {$q_3$};
\draw [black] (61.6,-19.3) circle (3);
\draw (61.6,-19.3) node {$S$};
\draw [black] (61.6,-19.3) circle (2.4);
\draw [black] (13.5,-19.2) -- (21.1,-19.2);
\fill [black] (21.1,-19.2) -- (20.3,-18.7) -- (20.3,-19.7);
\draw (17.3,-19.7) node [below] {$:,\rightarrow$};
\draw [black] (27.1,-19.23) -- (33,-19.27);
\fill [black] (33,-19.27) -- (32.2,-18.77) -- (32.2,-19.77);
\draw (30.05,-19.76) node [below] {$*,1$};
\draw [black] (22.777,-16.52) arc (234:-54:2.25);
\draw (24.1,-11.95) node [above] {$1,\rightarrow$};
\fill [black] (25.42,-16.52) -- (26.3,-16.17) -- (25.49,-15.58);
\draw [black] (34.677,-16.62) arc (234:-54:2.25);
\draw (36,-12.05) node [above] {$1,\rightarrow$};
\fill [black] (37.32,-16.62) -- (38.2,-16.27) -- (37.39,-15.68);
\draw [black] (39,-19.3) -- (45.4,-19.3);
\fill [black] (45.4,-19.3) -- (44.6,-18.8) -- (44.6,-19.8);
\draw (42.2,-19.8) node [below] {$\tvs,\leftarrow$};
\draw [black] (51.4,-19.3) -- (58.6,-19.3);
\fill [black] (58.6,-19.3) -- (57.8,-18.8) -- (57.8,-19.8);
\draw (55,-19.8) node [below] {$1,\tvs$};
\end{tikzpicture}
\end{center}
\end{frame}

\begin{frame}
\frametitle{Decision problems}
\begin{itemize}
\item A decision problem is a yes or no question. E.g:
\begin{itemize}
\item Given a graph $G$, is $G$ connected?
\item Does a given string of English characters contain the word \emph{biscuits} as a substring?
\item Is a given propositional formula satisfiable?
\item Is a given natural number prime?
\end{itemize}
\vspace{0.2cm}
\item A decision problem has a \emph{general form}, and an \emph{instance}. 
\vspace{0.2cm}
\item E.g. ``Is a given natural number prime?" is a general form, and a specific number $n$ is an instance of this problem.
\vspace{0.2cm}
\item Abstractly, a decision problem partitions its set of instances into two parts: one for `yes' and the other for `no'.
\vspace{0.2cm}
\item Turing machines can try to solve decision problems, but first we must write them in a language they can understand.
\end{itemize}
\end{frame}

\begin{frame}
\frametitle{Working with strings}
\begin{itemize}
\item A (finite) alphabet $\Sigma$ is a finite set of symbols.
\vspace{0.2cm}
\item A string over $\Sigma$ is a sequence of characters, e.g. 01001, or the digits of $\pi$.
\vspace{0.2cm}
\item $|s|$ denotes the length of $s$. E.g $|01001|=5$, and the length of the string defined by the digits of $\pi$ is $\omega$ (countable infinity).
\vspace{0.2cm}
\item The \emph{empty string} is denoted $\epsilon$, and $|\epsilon|=0$.
\vspace{0.2cm}
\item We can \emph{concatenate} finite strings. Given $a$ and $b$ we write $ab$.
\vspace{0.2cm}
\item For all finite strings $a$ we have $\epsilon a = a$ and $a\epsilon=a$.
\vspace{0.2cm}
\item $\Sigma^*$ is the set of all finite strings from $\Sigma$.
\end{itemize}
\end{frame}

\begin{frame}
\frametitle{Formal languages}
\begin{itemize}
\item A \emph{formal language} over $\Sigma$ is a subset of $\Sigma^*$.
\vspace{0.5cm}
\item If $L$ is a formal language over $\Sigma$ then we can define the \emph{characteristic function} of $L$ by $\chi_L:\Sigma^*\to\{0,1\}$ and $\chi_L(s) = \begin{cases}1 \text{ if } s\in L \\
0 \text{ if } s\notin L \end{cases}$  
\end{itemize}
\end{frame}

\begin{frame}
\frametitle{Encoding schemes}
\begin{itemize}
\item We want to take a decision problem and turn it into a language problem.
\vspace{0.2cm}
\item I.e. given a decision problem $D$ we want to associate it with a formal language $L_D$ for some alphabet $\Sigma$.
\vspace{0.2cm}
\item We have to choose $\Sigma$, and choose a system for writing the instances of $D$ as strings from $\Sigma^*$.
\vspace{0.2cm}
\item I.e. each instance $x$ should correspond to some $s_x\in\Sigma^*$. 
\vspace{0.2cm}
\item This system should be sensible. I.e:
\begin{itemize}
\item Different instances should be written as different strings.
\item We should be able to tell if a string from $\Sigma^*$ corresponds to an instance of $D$ or not.
\item We should be able to work out the string from the instance, and also the instance from the string.
\end{itemize}
\end{itemize}
\end{frame}

\begin{frame}
\frametitle{Encoding schemes - formal version}
\begin{definition}[encoding scheme]
If $\Sigma$ is a finite alphabet, $D$ is a decision problem, and $I_D$ is the set of all instances of $D$, an encoding scheme for $D$ using $\Sigma$ is a function $\mathbf{code}: I_D\to \Sigma^*$ such that:
\vspace{0.2cm}
\begin{enumerate}
\item if $x,y\in I_D$ and $x\neq y$ we must have $\mathbf{code}(x)\neq \mathbf{code}(y)$ (i.e. $\mathbf{code}$ is $1-1$),
\vspace{0.2cm}
\item it is possible to work out if a string in $\Sigma^*$ is $\mathbf{code}(x)$ for some $x\in I_D$, and
\vspace{0.2cm} 
\item the process for converting $x$ to $\mathbf{code}(x)$ and $\mathbf{code}(x)$ to $x$ is well defined. 
\end{enumerate}
\end{definition}
\end{frame}

\begin{frame}
\frametitle{Decision problems and formal languages}
\begin{itemize} 
\item Any formal language $L$ defines a decision problem $D_L$.
\vspace{0.2cm}
\begin{itemize}
\item ``is this word from $\Sigma^*$ in $L$?". 
\end{itemize}
\vspace{0.2cm}
\item Using encoding we can turn decision problems into formal languages as follows.
\end{itemize}

\begin{definition}[$L_D$]\label{D:LD}
Given a decision problem $D$, a finite alphabet $\Sigma$, and an encoding for $D$ using $\Sigma$, the language defined by $D$ is the set $L_D=\{\textbf{code}(x):x$ is a yes instance of $D\}$.
\end{definition}

\end{frame}

\begin{frame}
\frametitle{Decidability}

\begin{definition}[Decidable]\label{D:dec}
An (encodable) decision problem $D$ is decidable if there is a Turing machine $T$ that accepts when its input is the code of a yes instance of $D$, and rejects when its input is the code of a no instance (or not the code of an instance at all). We say $T$ decides $D$.
\end{definition}
\vspace{0.5cm}
\begin{itemize}
\item Is every decision problem that can be encoded using a finite alphabet decidable? 
\end{itemize}
\end{frame}

\begin{frame}
\frametitle{Semidecidability}
\begin{definition}[Semidecidable]\label{D:semidec}
An (encodable) decision problem $D$ is semidecidable if there is a Turing machine $T$ that accepts when its input is the code of a yes instance of $D$, and does not halt (i.e. it runs forever) when its input is the code of a no instance (or not the code of an instance at all). We say $T$ semidecides $D$.
\end{definition}  
\vspace{0.5cm}
\begin{itemize}
\item Is every decision problem that can be encoded using a finite alphabet semidecidable?
\vspace{0.2cm}
\item Why is every decidable problem also semidecidable? 
\end{itemize}
\end{frame}

\begin{frame}
\frametitle{Decidability and formal languages}
\begin{definition}[recursive]
A formal language $L$ over a finite alphabet $\Sigma$ is recursive if there is a Turing machine $T$ that takes input $x$ from $\Sigma^*$ and accepts when $x\in L$ and rejects when $x\notin L$.  
\end{definition}
\vspace{0.4cm}
\begin{definition}[recursively enumerable]
A formal language $L$ over a finite alphabet $\Sigma$ is recursively enumerable if there is a Turing machine $T$ that takes input $x$ from $\Sigma^*$ and accepts when $x\in L$ and runs forever when $x\notin L$. We often shorten \emph{recursively enumerable} to \emph{r.e.}).  
\end{definition}
\end{frame}

\begin{frame}
\frametitle{Some equivalences}
By definition of $L_D$ and $D_L$ we have the following correspondences for encodable decision problems:
\begin{align*}
D \text{ is decidable}&\iff L_D \text{ is recursive}\\
D_L \text{ is decidable}&\iff L \text{ is recursive}\\
D \text{ is semidecidable}&\iff L_D \text{ is recursively enumerable}\\
D_L \text{ is semidecidable}&\iff L \text{ is recursively enumerable}\\
\end{align*} 
\end{frame}

\begin{frame}
\frametitle{Example - palindromes}
\begin{center}
\resizebox{\columnwidth}{!}{%
\begin{tikzpicture}[scale=0.2]
\tikzstyle{every node}+=[inner sep=0pt]
\draw [black] (39.6,-15.5) circle (3);
\draw (39.6,-15.5) node {$q_1$};
\draw [black] (52.3,-15.5) circle (3);
\draw (52.3,-15.5) node {$B_0$};
\draw [black] (65.1,-15.5) circle (3);
\draw (65.1,-15.5) node {$B_1$};
\draw [black] (76.4,-15.5) circle (3);
\draw (76.4,-15.5) node {$B_2$};
\draw [black] (26.3,-15.5) circle (3);
\draw (26.3,-15.5) node {$A_0$};
\draw [black] (14.2,-15.5) circle (3);
\draw (14.2,-15.5) node {$A_1$};
\draw [black] (3.1,-15.5) circle (3);
\draw (3.1,-15.5) node {$A_2$};
\draw [black] (33.7,-25.1) circle (3);
\draw (33.7,-25.1) node {$q_0$};
\draw [black] (39.6,-6.2) circle (3);
\draw (39.6,-6.2) node {$R$};
\draw [black] (39.6,-51.4) circle (3);
\draw (39.6,-51.4) node {$F$};
\draw [black] (39.6,-51.4) circle (2.4);
\draw [black] (39.6,-37.7) circle (3);
\draw (39.6,-37.7) node {$S$};
\draw [black] (39.6,-37.7) circle (2.4);
\draw [black] (42.6,-15.5) -- (49.3,-15.5);
\fill [black] (49.3,-15.5) -- (48.5,-15) -- (48.5,-16);
\draw (45.95,-15) node [above] {$b,(\tvs,\ra)$};
\draw [black] (55.3,-15.5) -- (62.1,-15.5);
\fill [black] (62.1,-15.5) -- (61.3,-15) -- (61.3,-16);
\draw (58.7,-15) node [above] {$\{a,b\},\ra$};
\draw [black] (68.1,-15.5) -- (73.4,-15.5);
\fill [black] (73.4,-15.5) -- (72.6,-15) -- (72.6,-16);
\draw (70.75,-15) node [above] {$\tvs,\la$};
\draw [black] (36.6,-15.5) -- (29.3,-15.5);
\fill [black] (29.3,-15.5) -- (30.1,-16) -- (30.1,-15);
\draw (32.95,-15) node [above] {$a,(\tvs,\ra)$};
\draw [black] (23.3,-15.5) -- (17.2,-15.5);
\fill [black] (17.2,-15.5) -- (18,-16) -- (18,-15);
\draw (20.25,-15) node [above] {$\{a,b\},\ra$};
\draw [black] (11.2,-15.5) -- (6.1,-15.5);
\fill [black] (6.1,-15.5) -- (6.9,-16) -- (6.9,-15);
\draw (8.65,-15) node [above] {$\tvs,\la$};
\draw [black] (35.27,-22.54) -- (38.03,-18.06);
\fill [black] (38.03,-18.06) -- (37.18,-18.48) -- (38.04,-19);
\draw (36.02,-19.02) node [left] {$:,\ra$};
\draw [black] (5.614,-13.863) arc (121.48441:87.10459:54.115);
\fill [black] (36.61,-5.97) -- (35.84,-5.43) -- (35.79,-6.42);
\draw (17.18,-6.76) node [above] {$a,(\tvs,\la)$};
\draw [black] (42.592,-5.984) arc (92.59079:59.04384:55.895);
\fill [black] (42.59,-5.98) -- (43.41,-6.45) -- (43.37,-5.45);
\draw (62.17,-6.81) node [above] {$b,(\tvs,\la)$};
\draw [black] (66.423,-18.18) arc (54:-234:2.25);
\draw (65.1,-22.75) node [below] {$\{a,b\},\ra$};
\fill [black] (63.78,-18.18) -- (62.9,-18.53) -- (63.71,-19.12);
\draw [black] (15.523,-18.18) arc (54:-234:2.25);
\draw (14.2,-22.75) node [below] {$\{a,b\},\ra$};
\fill [black] (12.88,-18.18) -- (12,-18.53) -- (12.81,-19.12);
\draw [black] (38.277,-3.52) arc (234:-54:2.25);
\draw (39.6,1.05) node [above] {$\{a,b\},\la$};
\fill [black] (40.92,-3.52) -- (41.8,-3.17) -- (40.99,-2.58);
\draw [black] (39.6,-9.2) -- (39.6,-12.5);
\fill [black] (39.6,-12.5) -- (40.1,-11.7) -- (39.1,-11.7);
\draw (40.1,-10.85) node [right] {$\tvs,\ra$};
\draw [black] (36.714,-50.583) arc (-107.5108:-161.53957:50.565);
\fill [black] (36.71,-50.58) -- (36.1,-49.87) -- (35.8,-50.82);
\draw (14.7,-38.89) node [below] {$b,b$};
\draw [black] (75.335,-18.304) arc (-22.28191:-69.13662:57.811);
\fill [black] (42.43,-50.4) -- (43.36,-50.59) -- (43,-49.65);
\draw (63.87,-38.25) node [below] {$a,a$};
\draw [black] (36.842,-36.526) arc (-117.36188:-180.78654:20.004);
\fill [black] (36.84,-36.53) -- (36.36,-35.71) -- (35.9,-36.6);
\draw (28.23,-30.3) node [left] {$\tvs,\tvs$};
\draw [black] (52.825,-18.45) arc (5.36879:-64.91413:18.221);
\fill [black] (42.41,-36.66) -- (43.35,-36.77) -- (42.92,-35.86);
\draw (51.16,-30.43) node [right] {$\tvs,\tvs$};
\draw [black] (40.643,-18.311) arc (17.27171:-17.27171:27.917);
\fill [black] (40.64,-34.89) -- (41.36,-34.27) -- (40.4,-33.98);
\draw (42.4,-26.6) node [right] {$\tvs,\tvs$};
\end{tikzpicture}
}
\end{center}
\end{frame}

\begin{frame}
\frametitle{Pushdown automata}
\begin{itemize}
\item Turing machines are strictly more powerful than FSMs due to them effectively having infinite memory. 
\item There are things we can do with a TM that can not be done by an FSM. 
\item Is there a model of computation between FSMs and TMs? 
\item It turns out the answer is yes. 
\item A \emph{pushdown automaton} is another abstract computation system based on FSMs. 
\item A pushdown automaton also has access to infinite memory in the form of a stack. 
\item It turns out that this makes them strictly more powerful than FSMs, but strictly less powerful than Turing machines. 
\item Pushdown automata with two stacks turn out to be equivalent to Turing machines. 
\end{itemize}
\end{frame}

\begin{frame}
\frametitle{Computation and language}
\begin{definition}[Recognize]
Let $M$ be an abstract machine capable of acting on input from $\Sigma^*$ for a finite alphabet $\Sigma$, and let $L\subseteq\Sigma^*$ be a language. We say $M$ recognizes $L$ if $M$ accepts on input $x$ if and only if $x\in L$. 
\end{definition}
\begin{itemize}
\item Every recursively enumerable language has a Turing machine that recognizes it (this is just the definition of r.e.). 
\item Moreover, every Turing machine $T$ defines a recursively enumerable language (just take the set of all $x$ such that $T(x)$ halts). So...
\end{itemize}
\begin{theorem}
The class of recursively enumerable languages is precisely the class of languages that can be recognized by a Turing machine.
\end{theorem}
\end{frame}

\begin{frame}
\frametitle{The Chomsky heirarchy}
\begin{center}
\begin{tabular}{ |c| c | }
 \hline \textbf{language} & \textbf{model of computation} \\ \hline
 recursively enumerable & Turing machines \\ \hline 
 context-sensitive & linear bounded automata \\ \hline
 context-free & pushdown automata \\ \hline
 regular & finite state machines\\ \hline
\end{tabular}
\end{center}
\begin{itemize}
\item The Chomsky hierarchy of formal languages and their associated models of computation. 
\item Models of computation are arranged in order of power. 
\item On this course we are mainly interested in Turing machines. 
\item Note that it is possible to refine this hierarchy by adding extra classes, and this may result in something that is not linearly ordered.
\end{itemize}
\end{frame}





\end{document}