\documentclass[handout]{beamer} 
\title{ITCS 532:\\ Bonus question}
\date{}
\author{Rob Egrot}

\usepackage{amsmath, bbold, bussproofs,graphicx}
\usepackage{mathrsfs}
\usepackage{amsthm}
\usepackage{amssymb}
\usepackage[all]{xy}
\usepackage{multirow}
\usepackage{tikz-cd}


\newtheorem{proposition}[theorem]{Proposition}
\newcommand{\bN}{\mathbb{N}}
\newcommand{\bZ}{\mathbb{Z}}
\newcommand{\bQ}{\mathbb{Q}}
\newcommand{\bR}{\mathbb{R}}
\newcommand{\bP}{\mathbb{P}}
\newcommand{\tvs}{\textvisiblespace}
\newcommand{\ra}{\rightarrow}
\newcommand{\la}{\leftarrow}
\newcommand{\co}{\mathbf{code}}

\addtobeamertemplate{navigation symbols}{}{%
    \usebeamerfont{footline}%
    \usebeamercolor[fg]{footline}%
    \hspace{1em}%
    \insertframenumber/\inserttotalframenumber
}
\setbeamertemplate{theorems}[numbered]
\begin{document}

\begin{frame}
\titlepage
\end{frame}

\begin{frame}
Is $HAI$ r.e.?
\vspace{1cm}
\begin{itemize}
\item No.
\end{itemize}
\end{frame}

\begin{frame}
\begin{itemize}
\item Consider the complement of the halting problem $\bar{H}$.
\begin{itemize}
\item Yes instance is pair $(T,I)$ such that $T(I)$ does not halt.
\item No instance is $(T,I)$ such that $T(I)$ halts.
\end{itemize}
\vspace{1cm}
\item $\bar{H}$ is not r.e., as otherwise $H$ would be recursive (as $H$ is r.e.).
\end{itemize}
\end{frame}

\begin{frame}
\begin{itemize}
\item Given an instance $(T,I)$ of $\bar{H}$, define an instance $T_I$ of $HAI$.
\item $T_I(J)$ erases its input then simulates $T(I)$ for $|J|$ steps.
\item If $T(I)$ halts within $|J|$ steps then $T_I(J)$ loops.
\item If $T(I)$ does not halt within $|J|$ steps then $T_I(J)$ halts.
\end{itemize}
\begin{align*}(T,I) \text{ is yes of $\bar{H}$} &\implies T(I) \text{ does not halt}\\ &\implies T_I(J) \text{ halts for all } J \\&\implies T_I \text{ is yes of $HAI$.}\end{align*}
\begin{align*}(T,I) \text{ is no of $\bar{H}$} &\implies T(I) \text{ halts (in $n$ steps say)}\\
&\implies T_I(J) \text{ does not halt when } |J|>n\\
&\implies T_I \text{ is no of $HAI$}.\end{align*}
\begin{itemize}
\item So $\bar{H}\leq HAI$, and if $HAI$ were r.e. $\bar{H}$ would be too.
\end{itemize}
\end{frame}
\end{document}