\documentclass[handout]{beamer} 
\title{ITCS 532:\\ 
7. More on $p$-Time Reduction, and an Introduction to $\mathbf{NP}$}
\date{}
\author{Rob Egrot}

\usepackage{amsmath, bbold, bussproofs,graphicx}
\usepackage{mathrsfs}
\usepackage{amsthm}
\usepackage{amssymb}
\usepackage[all]{xy}
\usepackage{multirow}
\usepackage{tikz-cd}

\newtheorem{exercise}[theorem]{Exercise}{\bfseries}{\upshape}


\newtheorem{proposition}[theorem]{Proposition}
\newcommand{\bN}{\mathbb{N}}
\newcommand{\bZ}{\mathbb{Z}}
\newcommand{\bQ}{\mathbb{Q}}
\newcommand{\bR}{\mathbb{R}}
\newcommand{\bP}{\mathbb{P}}
\newcommand{\tvs}{\textvisiblespace}
\newcommand{\ra}{\rightarrow}
\newcommand{\la}{\leftarrow}
\newcommand{\co}{\mathbf{code}}

\newcommand{\Po}{\mathbf{P}}
\newcommand{\NP}{\mathbf{NP}}
\newcommand{\SAT}{\mathsf{PSAT}}

\addtobeamertemplate{navigation symbols}{}{%
    \usebeamerfont{footline}%
    \usebeamercolor[fg]{footline}%
    \hspace{1em}%
    \insertframenumber/\inserttotalframenumber
}
\setbeamertemplate{theorems}[numbered]
\begin{document}

\begin{frame}
\titlepage
\end{frame}


\begin{frame}
\frametitle{The Directed Hamiltonian Circuit Problem}
\begin{definition}[DHCP]
The directed Hamiltonian circuit problem ($DHCP$) asks whether a given finite \emph{directed} simple graph has a Hamiltonian circuit. A Hamiltonian circuit in a directed graph is like a Hamiltonian circuit in an undirected graph except we take the edge directions into account. 
\end{definition}
\end{frame}

\begin{frame}
\frametitle{$HCP\leq_p DHCP$}
\begin{theorem}
$HCP\leq_p DHCP$, ($p$-time in $|V|+|E|$).
\end{theorem}
\vspace{0.3cm}
\textbf{Proof}
\begin{itemize}
\item We start with a finite undirected simple graph $G=(V,E)$. 
\vspace{0.3cm}
\item Must construct in $p$-time a finite directed simple graph $G'$ that has a Hamiltonian circuit if and only if $G$ does. 
\vspace{0.3cm}
\item The vertices of $G'$ are copies of the vertices of $G$.
\vspace{0.3cm}
\item For every undirected edge in $G$ we add two directed edges to $G'$ between the same vertices (one in each direction). 
\vspace{0.3cm}
\item Construction of $G'$ is linear time ($O(m)$ if $m=|V|+|E|$). 
\end{itemize} 
\end{frame}

\begin{frame}
\frametitle{$HCP\leq_p DHCP$ - Checking the Reduction}
\begin{itemize}
\item $G$ is a yes instance if and only if it has a Hamiltonian circuit $v_1,v_2,\ldots,v_n$. 
\vspace{0.3cm}
\item If $v_1,v_2,\ldots,v_n$ is a Hamiltonian circuit in $G$ then there are directed edges in $G'$ from $v_i$ to $v_{i+1}$ for all $i=1,2,\ldots,n-1$, and also from $v_n$ to $v_1$. 
\vspace{0.3cm}
\item So $v_1,v_2,\ldots,v_n$ is also a Hamiltonian circuit in $G'$. 
\vspace{0.3cm}
\item Conversely, if $v_1,v_2,\ldots,v_n$ is a Hamiltonian circuit in $G'$ then there must be edges in $G$ connecting $v_i$ and $v_{i+1}$ for all $i=1,2,\ldots,n-1$, and also an edge connecting $v_n$ and $v_1$. 
\vspace{0.3cm}
\item So $v_1,v_2,\ldots,v_n$ is also a Hamiltonian circuit in $G$ as required.    
\end{itemize} 
\end{frame}

\begin{frame}
\frametitle{$DHCP\leq_p HCP$}

\begin{theorem}
$DHCP\leq_p HCP$ ($p$-time in $|V|+|E|$).
\end{theorem}
\vspace{0.3cm}
\textbf{Proof}
\begin{itemize}
\item This direction is not easy. 
\vspace{0.3cm}
\item We start with a directed simple graph $G$ and we must construct in $p$-time an undirected  simple graph $G'$ that preserves `yes' and `no' instances. 
\vspace{0.3cm}
\item This is hard because it's not at all obvious how to encode the direction of edges of $G$ in $G'$. 
\vspace{0.3cm}
\item This is a problem because a Hamiltonian circuit in $G$ must travel along edges in the right direction. 
\vspace{0.3cm}
\item If we lose information about the direction of edges then Hamiltonian circuits in $G'$ may not correspond to Hamiltonian circuits in $G$. 
\vspace{0.3cm}
\item Fortunately, with some thought we can overcome this problem.  
\end{itemize} 
\end{frame}

\begin{frame}
\frametitle{$DHCP\leq_p HCP$ - The Construction of $G'$}
\begin{itemize}
\item If $V$ and $E$ are the sets of vertices and edges of $G$ respectively, we define $G'=(V',E')$ as follows:
\begin{itemize}
\item $V'=\bigcup\{\{v,v^I,v^O\}:v\in V\}$
\item $E'=\{\{v_0^O,v_1^I\} : (v_0,v_1)\in E\} \cup \{\{v,v^I\},\{v,v^O\}:v\in V\}$
\end{itemize}

\item So every vertex $v$ of $G$ gets two partner vertices, $v^I$ and $v^O$, that encode the direction of edges. 
\item There are edges in $G'$ connecting $v$ to $v^O$ and $v^I$, and for a directed edge $(v,v')$ in $G$ there's an edge between $v^O$ and $v'^I$ in $G'$. E.g.
\[\xymatrix@R=0.5em{
\ar[dr]& & &\\
& \bullet_{v_0}\ar[r] &\bullet_{v_1}\ar[ur]\ar[dr] & \\
\ar[ur] & & &}\]
becomes 
\[\xymatrix@R=0.5em{
\ar@{-}[dr] & & & & & & & &\\
& \bullet_{v_0^I}\ar@{-}[r] & \bullet_{v_0}\ar@{-}[r] & \bullet_{v_0^O}\ar@{-}[r] & \bullet_{v_1^I}\ar@{-}[r] & \bullet_{v_1}\ar@{-}[r] & \bullet_{v_1^O}\ar@{-}[ur]\ar@{-}[dr]\\
\ar@{-}[ur] & & & & & & & &}\]
\end{itemize} 
\end{frame}

\begin{frame}
\frametitle{$DHCP\leq_p HCP$ - Is the Reduction Correct?}
\begin{itemize}
\item We need to show $G$ has an HC if and only if $G'$ does. 
\vspace{0.5cm}
\item If $|V|=n$ and $v_1,v_2,\ldots,v_n$ is a Hamiltonian circuit in $G$, then\vspace{0.5cm} 
\begin{equation*}v^I_1,v_1, v^O_1, v_2^I, v_2, v_2^O,\ldots,v_n^I,v_n,v_n^O\end{equation*} 
\vspace{0.5cm}
is a Hamiltonian circuit in $G'$. 
\end{itemize} 
\end{frame}

\begin{frame}
\frametitle{$DHCP\leq_p HCP$ - Is the Reduction Correct?}
\begin{itemize}
\item Conversely, if $|V|=n$ then $|V'|=3n$. 
\item Suppose $c=u_1,u_2,\ldots,u_{3n}$ is a Hamiltonian circuit in $G'$. 
\item Then every $v\in V$ appears somewhere in $c$. 
\item For each $v\in V$, the edges to $v$ are $\{v,v^I\}$ and $\{v,v^O\}$. 
\item Since $|V|=n$ and $c$ has exactly $3n$ elements, a simple counting argument tells us that, after choosing a suitable starting point, $c$ must have form either  \begin{equation*}c= v^I_1,v_1, v^O_1, v_2^I, v_2, v_2^O,\ldots,v_n^I,v_n,v_n^O,\end{equation*} or \begin{equation*}c=v^O_1,v_1, v^I_1, v_2^O, v_2, v_2^I,\ldots,v_n^O,v_n,v_n^I.\end{equation*} 
\item So by the construction of $G'$, we get a Hamiltonian circuit in $G$ by either following $c$ forwards or backwards.
\end{itemize} 
\end{frame}

\begin{frame}
\frametitle{$DHCP\leq_p HCP$ - Is the Reduction $p$-time?}
\begin{itemize}
\item We still have to check the reduction is $p$-time in $|V|+|E|$. 
\vspace{0.5cm}
\item First we constructed the vertices of $G'$. 
\vspace{0.5cm}
\item There are three of these for every vertex of $G$, so this step runs in $O(|V|)$ time. 
\vspace{0.5cm}
\item Then we added edges of form $(v^I,v)$ and $(v,v^O)$ for every vertex of $G$. Again this is $O(|V|)$.
\vspace{0.5cm}
\item Finally we added edges of form $(u^O,v^I)$ for every edge of $G$. This is obviously $O(|E|)$, to so the combined running time is also $O(|V|+|E|)$.
\end{itemize} 
\end{frame}

\begin{frame}
\frametitle{Verification vs Construction}
\begin{itemize}
\item Every natural number can be written as a product of prime numbers in an essentially unique way. 
\item This is the \emph{Fundamental Theorem of Arithmetic}. 
\item Given $n\in\bN$, an obvious question is, `what is the prime decomposition of $n$?'. 
\item This is not, in general, an easy question to answer. 
\item In fact, no $p$-time algorithm is known. 
\item It \emph{is} easy to check whether a proposed factorization is correct. 
\item This illustrates an important principle: 
\begin{itemize}
\item It can be easy to \emph{check} a solution, but hard to \emph{find} a solution.
\end{itemize}
\item In the case of the prime factorization problem, this asymmetry lies at the heart of RSA encryption.
\item We can formalize this in terms of decision problems and Turing machines with something called a \emph{verifier}.
\end{itemize} 
\end{frame}

\begin{frame}
\frametitle{Verifiers}
\begin{definition}[Verifier]
Given a language $L$ over a finite alphabet $\Sigma$, a verifier for $L$ is a modified Turing machine $V$, where $V$ takes two inputs instead of one (consider the inputs to be separated by a special symbol), and such that
\[x\in L\iff \text{ there is } y\in\Sigma^* \text{ such that } V(x,y) \text{ accepts}\] 
\end{definition}
\begin{itemize}
\item Here $y$ is called a \emph{certificate} for $x$. 
\item A verifier must halt for all inputs. 
\item We say a verifier for $L$ runs in $p$-time if it runs in $p$-time with respect to the length of $x$ for all inputs $(x,y)$. 
\item The certificate is like a proposed solution. 
\item The verifier accepts $x$, which encodes an instance of a decision problem, if and only if there's some proposed solution $y$ that it can check is actually a solution. 
\end{itemize}
\end{frame}

\begin{frame}
\frametitle{Example}
\begin{example}[Composite numbers]
\begin{itemize}
\item We could build a verifier to check if numbers are composite (i.e. not prime). 
\item Here the certificates would be potential non-trivial divisors. 
\item $V$ would run by checking whether $y$ non-trivially divides $x$ (so we exclude $y=1$ or $y=x$). 
\item If $y\mid x$ and $y\notin\{1,x\}$ then $V(\co(x),\co(y))$ accepts. 
\item If not it rejects.
\end{itemize}
\end{example} 
\end{frame}

\begin{frame}
\frametitle{Example}
\begin{example}[Hamiltonian circuits]
\begin{itemize}
\item An instance of the Hamiltonian circuit problem is a graph $G$. 
\item A certificate for $G$ is just a sequence $s$ of vertices (in coded form). 
\item A verifier $V$ for the Hamiltonian  circuit problem would accept on input $(\co(G),\co(s))$ if and only if $s$ defines a Hamiltonian circuit in $G$.
\end{itemize} 
\end{example}
\end{frame}

\begin{frame}
\frametitle{What's the Point of Verifiers?}
\begin{itemize}
\item If we don't care about run time, then verifiers don't really add anything to our toolkit for describing languages.
\vspace{0.5cm}
\item The next theorem demonstrates this. 
\vspace{0.5cm}
\item However, when we do care about run times this changes, as we will see.
\end{itemize}
\end{frame}

\begin{frame}
\frametitle{Verifiers and R.E. languages}
\begin{theorem}
A language has a verifier if and only if it is recursively enumerable.
\end{theorem}
\textbf{Proof}
\begin{itemize}
\item Let $L$ be r.e. and let $T$ be a TM that semidecides $L$. 
\item We construct a verifier $V_T$ so that when $V_T$ runs on $(x,\co(n))$ it computes $T(x)$ for $n$ steps ($n\in \bN$). 
\item It accepts if $T(x)$ accepts within $n$ steps, and rejects otherwise. 
\item It also rejects if $y$ is not $\co(n)$ for some $n$.
\item $V_T$ is a verifier for $L$ because: 
\begin{itemize}
\item If $x\in L$ there is $n$ such that $T(x)$ halts in $n$ steps.
\item So $V_T(x,\co(n))$ accepts.
\item I.e. strings in $L$ have a certificate.
\item $V_T(x,y)$ rejects if $y$ is not $\co(n)$ for any $n$.
\item $V_T(x,\co(n))$ rejects if $T(x)$ does not halt within $n$ steps.
\item I.e. strings not in $L$ have no certificate.
\end{itemize}
\end{itemize} 
\end{frame}

\begin{frame}
\frametitle{Verifiers and R.E. languages}
\textbf{Proof Continued}
\begin{itemize}
\item For the converse, let $L$ have a verifier $V$. 
\vspace{0.5cm}
\item We construct a Turing machine $T_V$ such that $T_V(x)$ operates by generating strings $y$ in lexicographic order and computing $V(x,y)$, and halting if $V(x,y)$ accepts. 
\vspace{0.5cm}
\item Then $T_V$ semidecides $L$ because $x\in L$ if and only if there is $y$ such that $V(x,y)$ accepts, and, as $T_V$ computes $V(x,y)$ for every $y$ eventually, it will halt if and only if $x\in L$.  
\end{itemize} 
\end{frame}

\begin{frame}
\frametitle{Complexity Classes}
\begin{itemize}
\item  We say that languages/decision problems are in the complexity class $\Po$ if there is a Turing machine that decides them in $p$-time. 
\vspace{0.3cm}
\item We say languages/decision problems are in the complexity class $\NP$ if they have a verifier that runs in $p$-time. 
\vspace{0.3cm}
\item The distinction between $\Po$ and $\NP$ formalizes the distinction between problems that can be efficiently solved, and problems whose solutions can be efficiently checked. 
\vspace{0.3cm}
\item As is often the case in complexity theory, these classes are rather coarse. 
\vspace{0.3cm}
\item There's a big difference between a problem with a linear time solution, and a problem with an $O(n^{10,000,000})$ solution, but we gloss over it.
\end{itemize} 
\end{frame}

\begin{frame}
\frametitle{Turing Machine Variants}
\begin{itemize}
\item  We have defined $\Po$ and $\NP$ in terms of kinds of Turing machines, and, as we have seen, there are many variant definitions we could use (e.g. multi-tape etc.). 
\vspace{0.3cm}
\item These are equivalent to the `standard' definition in terms of what problems they can decide. 
\vspace{0.3cm}
\item But could using a different model of computation affect which problems are in $\Po$ and $\NP$. 
\vspace{0.3cm}
\item For example, we know a problem that can be solved by a `standard' Turing machine could likely be solved in fewer steps by a Turing machine with more tapes. 
\vspace{0.3cm}
\item Could this speed increase be such that there is a problem solvable in $p$-time by a multi-tape machine that is not solvable in $p$-time by a `standard' machine? 
\end{itemize} 
\end{frame}

\begin{frame}
\frametitle{Turing Machine Variants}
\begin{itemize}
\item  The answer turns out to be no.
\vspace{0.3cm}
\item If we carefully analyze the modeling of multi-tape Turing machines by single-tape Turing machines from a previous class, we see that the number of steps taken is polynomially bounded by the number of steps the multi-tape machine takes. 
\vspace{0.3cm}
\item Therefore, a problem that is $p$-time for a multi-tape machine will also be $p$-time for a single tape machine. 
\vspace{0.3cm}
\item The same is true for the other Turing machine variants too. 
\vspace{0.3cm}
\item This means our definitions for $\Po$ and $\NP$ are \emph{robust}. 
\vspace{0.3cm}
\item They do not depend unreasonably on the definition of `Turing machine'.     
\end{itemize} 
\end{frame}

\begin{frame}
\frametitle{The Class $\NP$}
\begin{itemize}   
\item Note that $\NP$ does \emph{not} stand for \emph{non-polynomial}. 
\vspace{1cm}
\item This is a common but understandable misconception. 
\vspace{1cm}
\item $\NP$ actually stands for \emph{non-deterministic polynomial}. 
\vspace{1cm}
\item The name comes from an alternative definition of the class $\NP$, which we will see in the next class.
\end{itemize} 
\end{frame}

\end{document}