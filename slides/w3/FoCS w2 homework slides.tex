\documentclass[handout]{beamer} 
\title{ITCS 532:\\ W2 Homework Solutions}
\date{}
\author{Rob Egrot}

\usepackage{amsmath, bbold, bussproofs,graphicx}
\usepackage{mathrsfs}
\usepackage{amsthm}
\usepackage{amssymb}
\usepackage[all]{xy}
\usepackage{multirow}
\usepackage{tikz-cd}


\newtheorem{proposition}[theorem]{Proposition}
\newcommand{\bN}{\mathbb{N}}
\newcommand{\bZ}{\mathbb{Z}}
\newcommand{\bQ}{\mathbb{Q}}
\newcommand{\bR}{\mathbb{R}}
\newcommand{\bP}{\mathbb{P}}
\newcommand{\tvs}{\textvisiblespace}
\newcommand{\ra}{\rightarrow}
\newcommand{\la}{\leftarrow}

\addtobeamertemplate{navigation symbols}{}{%
    \usebeamerfont{footline}%
    \usebeamercolor[fg]{footline}%
    \hspace{1em}%
    \insertframenumber/\inserttotalframenumber
}
\setbeamertemplate{theorems}[numbered]
\begin{document}

\begin{frame}
\titlepage
\end{frame}

\begin{frame}
\frametitle{Q1}
Let $\Sigma=\{a,b,c\}$. Let $L$ be the set of all finite strings containing the substring $ab$. Design a Turing machine that decides $L$. What regular expression describes $L$?
\begin{center}
\begin{tikzpicture}[scale=0.2]
\tikzstyle{every node}+=[inner sep=0pt]
\draw [black] (15.4,-24.4) circle (3);
\draw (15.4,-24.4) node {$q_0$};
\draw [black] (26.9,-24.4) circle (3);
\draw (26.9,-24.4) node {$a$};
\draw [black] (39.3,-24.4) circle (3);
\draw (39.3,-24.4) node {$b$};
\draw [black] (50.4,-24.4) circle (3);
\draw (50.4,-24.4) node {$A$};
\draw [black] (50.4,-24.4) circle (2.4);
\draw [black] (26.9,-37.1) circle (3);
\draw (26.9,-37.1) node {$R$};
\draw [black] (26.9,-37.1) circle (2.4);
\draw [black] (18.4,-24.4) -- (23.9,-24.4);
\fill [black] (23.9,-24.4) -- (23.1,-23.9) -- (23.1,-24.9);
\draw (21.15,-24.9) node [below] {$:,\ra$};
\draw [black] (29.478,-22.889) arc (111.61948:68.38052:9.829);
\fill [black] (36.72,-22.89) -- (36.16,-22.13) -- (35.79,-23.06);
\draw (33.1,-21.7) node [above] {$a,\ra$};
\draw [black] (25.577,-21.72) arc (234:-54:2.25);
\draw (26.9,-17.15) node [above] {$\{b,c\},\ra$};
\fill [black] (28.22,-21.72) -- (29.1,-21.37) -- (28.29,-20.78);
\draw [black] (26.9,-27.4) -- (26.9,-34.1);
\fill [black] (26.9,-34.1) -- (27.4,-33.3) -- (26.4,-33.3);
\draw (26.4,-30.75) node [left] {$\tvs,\tvs$};
\draw [black] (37.2,-26.55) -- (29,-34.95);
\fill [black] (29,-34.95) -- (29.91,-34.73) -- (29.2,-34.03);
\draw (33.63,-32.22) node [right] {$\tvs,\tvs$};
\draw [black] (42.3,-24.4) -- (47.4,-24.4);
\fill [black] (47.4,-24.4) -- (46.6,-23.9) -- (46.6,-24.9);
\draw (44.85,-24.9) node [below] {$b,b$};
\draw [black] (37.977,-21.72) arc (234:-54:2.25);
\draw (39.3,-17.15) node [above] {$a,\ra$};
\fill [black] (40.62,-21.72) -- (41.5,-21.37) -- (40.69,-20.78);
\draw [black] (36.558,-25.599) arc (-73.45716:-106.54284:12.146);
\fill [black] (29.64,-25.6) -- (30.27,-26.31) -- (30.55,-25.35);
\draw (33.1,-26.6) node [below] {$c,\ra$};
\end{tikzpicture}
\end{center}

Regular expression: $\cdot ab \cdot$ or $(a|b|c)^\ast ab (a|b|c)^\ast$.
\end{frame}

\begin{frame}
\frametitle{Q2}
Let $\Sigma$ be a finite alphabet. Consider the class $\mathcal C$ of Turing machine variants over $\Sigma$ with a countably infinite number of tape heads instead of only one. So the transition function is defined by \begin{equation*}\delta:(Q\setminus H)\times \Sigma^\omega \to Q\times(\Sigma\setminus\{:\}\cup \{\la,\ra\})^\omega \end{equation*}
Let $L\subseteq \Sigma^*$ be any language over $\Sigma$. Show that there is a machine in $\mathcal C$ that decides $L$. What does this tell us about the relative power of $\mathcal C$ and the class of regular Turing machines?
\end{frame}

\begin{frame}
\frametitle{Q2 - solution}
If we assume the machine can tell when two or more heads are reading the same space, we can design a machine as follows:
\begin{enumerate}
\item Move all the tape heads to the right, so they are all reading the first symbol on the tape.
\item Move all tape heads except the first to the right, so the first is still reading the first symbol, and the rest are reading the second symbol. 
\item Move all except the first and second tape head to the right.
\item Keep following this pattern till the first blank space is found.
\item Now there are $n$ tape heads reading the $n$ characters of the input string, and the rest of the tape heads are all reading a blank space.
\item Write the transition function so that if the string being read is in $L$ the machine accepts, and it rejects otherwise.
\end{enumerate}
This tells us this new kind of machine is much more powerful than regular Turing machines. 
\end{frame}

\begin{frame}
\frametitle{Q2 - better solution}
Without the assumption:
\begin{itemize}
\item (Step 0) Move all the tape heads to the right, so they are all reading the first symbol on the tape.
\item (Step 1) Move all the tape heads $h_i$ such that $2|i$ to the right, and keep the others in place, except for $h_1$ which moves back to the start of the tape.
\item (Step 2) Move all the tape heads $h_i$ such that $2$ and $3$ divide $i$ to the right. Keep the others in place except $h_1$ which moves back to square 1 (as it must do this), and also move $h_3$ back to the start of the tape.
\item (Step 3) Move all the tape heads $h_i$ such that $i$ is divisible by $2,3,5$ to the right. Keep the others in place except $h_3$ which moves back to square 1, and also move $h_5$ back to the start of the tape.
\end{itemize}
\end{frame}

\begin{frame}
\frametitle{Q2 - better solution continued}
\begin{itemize}
\item (Step $k$) Suppose primes are $p_1,p_2,\ldots$. At the start of computation step $k>2$, the tape heads with numbers divisible by $p_1\times\ldots \times p_{k-1}$ are reading cell $k-1$, and $h_{p_{k-1}}$ is reading the $:$ symbol.  

During step $k$ all tape heads with numbers divisible by $p_1\times\ldots \times p_{k}$ advance to the right, and head $h_{p_k}$ goes back to read $:$.
\item At some point there will be an infinite number of tape heads reading the blank space symbol. This tells the machine the end of the input has been reached.
\item Write the transition function so that if the string being read is in $L$ the machine accepts, and it rejects otherwise.
\end{itemize}
\end{frame}

\begin{frame}
\frametitle{Q3}
Let $\Sigma$ be a finite alphabet. Prove that $\Sigma^*$ is countably infinite. 
\vspace{0.5cm}
\begin{itemize}
\item $\Sigma^*$ is the set of all finite strings using $\Sigma$, so is infinite. 
\vspace{0.2cm}
\item To prove it is countable it is sufficient to find a 1-1 function $f:\Sigma^*\to \bN$. 
\vspace{0.2cm}
\item Suppose $\Sigma = \{\sigma_1,\ldots,\sigma_n\}$. 
\vspace{0.2cm}
\item Let $p_1,p_2,p_3,\ldots$ list the prime numbers in ascending order. 
\vspace{0.2cm}
\item Given a string $s = \sigma_{i_1}\sigma_{i_2}\ldots\sigma_{i_k}$, define $f(s) = p_1^{i_1}\times p_2^{i_2}\times \ldots \times p_k^{i_k}$. 
\vspace{0.2cm}
\item E.g. if $\Sigma = \{ a = \sigma_1, b = \sigma_2, c=\sigma_3\}$, the string $aacb$ corresponds to $2^13^15^37^2$.  
\vspace{0.2cm}
\item Then $f$ is 1-1 because, by the Fundamental Theorem of Arithmetic, numbers are specified uniquely by their prime factorizations (up to reordering).
\end{itemize}
\end{frame}

\begin{frame}
\frametitle{Q4}
Let $\Sigma$ again be a finite alphabet. Is $\wp(\Sigma^*)$ countable? Justify your answer.
\vspace{0.7cm}
\begin{itemize}
\item No (unless $\Sigma$ is empty).
\vspace{0.3cm}
\item As proved in the notes for the 531 course, given a non-empty set $X$ we always have $|X|<|\wp(X)|$. 
\end{itemize}
\end{frame}

\begin{frame}
\frametitle{Q5}
Recall that the union of a countable number of countable sets is countable. Let $\Sigma=\{0,1\}$ and let $X=\wp(\Sigma^*)$ be the uncountable set of all languages over $\Sigma$. Let $C_1$ and $C_2$ be countable subsets of $X$, and let $U$ be an uncountable subset of $X$. Remember that if $Y$ is a set we use $\bar{Y}$ to denote the complement of $Y$. For each of the following sets say whether it is countable, uncountable, or dependent on the choice of $C_1,C_2,U$. Justify your answers.
\begin{enumerate}[(a)]
\item$C_1\cup C_2$
\item$\bar{C_1}$
\item$\bar{U}$
\item$\bar{C_1}\cap U$
\end{enumerate} 
\end{frame}

\begin{frame}
\frametitle{Q5 - solutions (a),(b)}
\vspace{0.7cm}
\begin{enumerate}[(a)]
\item ($C_1\cup C_2$) Union of a countable number of countable sets is countable.
\vspace{0.5cm}
\item ($\bar{C_1}$) Uncountable.
\vspace{0.3cm}
\begin{itemize}
\item   $C_1\cup \bar{C}_1 = X$, so if both $C_1$ and $\bar{C}_1$ were countable then $X$ would be too.
\vspace{0.3cm}
\item But $X$ is uncountable. 
\vspace{0.3cm}
\item So, since $C_1$ is countable, $\bar{C}_1$ must be uncountable.
\end{itemize} 
\end{enumerate}
\end{frame}

\begin{frame}
\frametitle{Q5 - solutions (c)}
\begin{enumerate}[]
\item[(c)] ($\bar{U}$) This depends on the choice of $U$. 
\vspace{0.7cm}
\begin{itemize}
\item If $U = X$ then $\bar{U} = \emptyset$, which is countable. 
\vspace{0.3cm}
\item Let $U$ be the set of all languages containing the string $s$, for some arbitrary choice of $s$. 
\vspace{0.3cm}
\item Then $U$ is in bijection with $\wp(\Sigma^*\setminus\{s\})$. 
\vspace{0.3cm}
\item Bijection takes $L\in U$ to $L\setminus\{s\}$, which is in $\wp(\Sigma^*\setminus\{s\})$, and conversely takes $L'\in\wp(\Sigma^*\setminus\{s\})$ to $L'\cup\{s\}$, which is in $U$.
\vspace{0.3cm} 
\item Moreover, $\bar{U} = \wp(\Sigma^*\setminus\{s\})$. 
\vspace{0.3cm}
\item $\wp(\Sigma^*\setminus\{s\})$ is uncountable. 
\end{itemize} 
\end{enumerate}
\end{frame}

\begin{frame}
\frametitle{Q5 - solutions (d)}
\begin{enumerate}[]
\item[(d)] ($\bar{C_1}\cap U$) This is uncountable, for the following reason.
\begin{align*}
U &= X \cap U \\
&=(\bar{C}_1\cup C_1)\cap U \\
&= (\bar{C}_1 \cap U) \cup (C_1\cap U).
\end{align*}

Now, $(C_1\cap U)$ is countable, so $(\bar{C}_1 \cap U)$ must be uncountable as $U$ is. 
\end{enumerate}
\end{frame}



\end{document}