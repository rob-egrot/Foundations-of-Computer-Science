\documentclass{article}

\usepackage{amsmath, mathrsfs, amssymb, stmaryrd, cancel, relsize,tikz,amsthm,comment,enumerate}

\theoremstyle{definition}
\newtheorem{Q}{Question}
\newtheorem{definition}{Definition}

\newcommand{\tvs}{\textvisiblespace}
\newcommand{\ra}{\rightarrow}
\newcommand{\la}{\leftarrow}
\newcommand{\co}{\mathbf{code}}

%\includecomment{comment}


\title{ITCS 532 Foundations of Computer Science\\
Week 6 - Tractability and $p$-Time Reduction (Homework)}
\author{Rob Egrot}
\date{}

\begin{document}
\maketitle

\begin{Q}
Prove that $\equiv_p$ is an equivalence relation.
\end{Q}
\begin{comment}
\textbf{Solution}
We know $\leq_p$ is reflexive by exercise 7.5.1 in the notes, so clearly $A\equiv_p A$. We also know $\leq_p$ is transitive by theorem 7.5.2, so if $A\equiv_p B$ and $B\equiv_p C$, then by definition of $\equiv_p$ we have $A\leq_p B$ and $B\leq_p C$, so $A\leq_p C$. Similarly we have $C\leq_p A$ and so $A\equiv_p C$. So $\equiv_p$ is transitive. Finally, $A\equiv_p B \iff A\leq_p B$ and $B\leq_p A$, and so $\equiv_p$ is obviously symmetric.
\end{comment}

\begin{Q}
Let $k\in \mathbb{N}\setminus\{0\}$.
\begin{enumerate}[a)]
\item Prove that $k^{n+1}$ is $O(k^n)$ as a function of $n$.
\item Prove that $k^{2n}$ is not $O(k^n)$ as a function of $n$.
\end{enumerate}
\end{Q}
\begin{comment}
\textbf{Solution}
\begin{enumerate}[a)]
\item We have $k^{n+1} = k.k^n $ for all $n$. So $k^{n+1}\leq ck^n$ when $c=k$. I.e. $k^{n+1}$ is $O(k^n)$.
\item Suppose $k^{2n}$ is $O(k^n)$. Then there is a constant $c$ with $k^{2n}\leq ck^n$ for `large' $n$. Taking $\log_k$ of both sides gives
\[2n\leq \log_k c + n,\]
and so
\[n \leq \log_k c,\]
but this obviously cannot be true for all `large' $n$. 

\end{enumerate}
\end{comment}

\begin{Q}
Let $C_1$ and $C_2$ be computers, and suppose that $C_2$ is $2^9$ times faster than $C_1$. I.e. if $C_1$ can take $v_1$ computation steps per second then $C_2$ can take $v_2=2^9v_1$ computation steps per second. $C_1$ and $C_2$ both run the same $O(n^3)$ algorithm for sorting a list of numbers of size $n$. Suppose the largest list size that $C_1$ can guarantee to sort in some fixed time $t$ is 1000. I.e. $C_1$ can guarantee to sort a list with size at most 1000 in time $t$ (a whole number of seconds), but larger lists may take longer. Approximately what is the largest list size that $C_2$ is guaranteed to sort in the same time $t$?  
\end{Q}
\begin{comment}
\textbf{Solution}
Let $n_i$ be the largest list size that $C_i$ is guaranteed to sort within time $t$, for $i\in\{1,2\}$. Since the algorithm is $O(n^3)$, there is a constant $c$ such that the run time on a list of length $n$ is at most $cn^3$. The number of computation steps $C_i$ can take in time $t$ is $v_it$. So $n_i = \max \{n : cn^3\leq v_it\}$. I.e. $n_1 = \lfloor (\frac{v_1t}{c})^{\frac{1}{3}} \rfloor$, and $n_2 = \lfloor (\frac{v_2t}{c})^{\frac{1}{3}} \rfloor$. But $v_2 = 2^9v_1$, and so $n_2 = \lfloor (\frac{2^9v_1t}{c})^{\frac{1}{3}} \rfloor \approx 2^3\lfloor (\frac{v_1t}{c})^{\frac{1}{3}} \rfloor = 2^3n_1$. We are told $n_1 = 1000$, so $n_2\approx 2^3\times 1000 = 8000$. 

  
\end{comment}

\begin{Q}
\emph{This question is to review the technique of reduction (not $p$-time reduction, just ordinary reduction).} Let $\Sigma = \{0,1\}$. Let $D$ be the decision problem ``Given a Turing machine $T$ over $\Sigma$, does $T(I)$ halt whenever $|I|$ is odd?". 
\begin{enumerate}[a)]
\item Define the empty tape halting problem ($ETHP$).
\item Show that $ETHP \leq D$.
\item Is $D$ decidable? Justify your answer.
\end{enumerate}
\end{Q}
\begin{comment}
\textbf{Solution}
\begin{enumerate}[a)]
\item ``Given a Turing machine $T$, does $T$ halt when run on a blank tape?''.
\item An instance of $ETHP$ is a Turing machine $T$. An instance of $D$ is also a Turing machine. Given a Turing machine $T$ we will construct $T'$ such that $T$ halts on the empty input iff $T'(I)$ halts whenever $|I|$ is odd. $T'$ operates on an input $I$ as follows:
\begin{enumerate}[1)]
\item First $T'$ erases $I$ and moves the tape head back to the start of the tape.
\item $T'$ then does what $T$ do (note the tape is now empty).
\end{enumerate}
Then $T(\epsilon)$ halts implies $T'(I)$ halts for all $I$, so also whenever $|I|$ is odd. Conversely, $T'$ does the same thing for all inputs, and $T'(I)$ halts for all $I$ with $|I|$ odd implies $T(\epsilon)$ halts.
\item We conclude that $D$ is undecidable. As $ETHP\leq D$, an algorithm solving $D$ would imply an algorithm solving $ETHP$, which is impossible, as $ETHP$ is undecidable.
\end{enumerate}
\end{comment}

\end{document}